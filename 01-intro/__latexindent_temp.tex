%%%%%%%%%%%%%%%%%%%%%%%%%%%%%%%%%%%%%%%%%%%%%%%%%%%%%%%%%%%%%%%%%%%%%%%%%%%%%%%%
% TUM-Vorlage: Präsentation
%%%%%%%%%%%%%%%%%%%%%%%%%%%%%%%%%%%%%%%%%%%%%%%%%%%%%%%%%%%%%%%%%%%%%%%%%%%%%%%%
%
% Rechteinhaber:
%     Technische Universität München
%     https://www.tum.de
% 
% Gestaltung:
%     ediundsepp Gestaltungsgesellschaft, München
%     http://www.ediundsepp.de
% 
% Technische Umsetzung:
%     eWorks GmbH, Frankfurt am Main
%     http://www.eworks.de
%
%%%%%%%%%%%%%%%%%%%%%%%%%%%%%%%%%%%%%%%%%%%%%%%%%%%%%%%%%%%%%%%%%%%%%%%%%%%%%%%%


%%%%%%%%%%%%%%%%%%%%%%%%%%%%%%%%%%%%%%%%%%%%%%%%%%%%%%%%%%%%%%%%%%%%%%%%%%%%%%%%
% Zur Wahl des Seitenverhältnisses bitte einen der beiden folgenden Befehle
% auskommentieren und den ausführen lassen:
%\documentclass[aspectratio=169]{beamer}
\documentclass[t,aspectratio=169]{beamer}
\usepackage[
	orientation=landscape,
	size=custom,
	width=25.4,
	height=14.2875,
	scale=0.5
]{beamerposter}
\usepackage{verbatim}
\usepackage{booktabs}
\usepackage{verbatim}
\usepackage[cache=false]{minted}
\usemintedstyle{tango}
\definecolor{codeBG}{rgb}{0.97, 0.97, 0.99}
\setminted{linenos=true, bgcolor=codeBG, mathescape=true}

\newcommand{\PraesentationSchriftgroesseSehrGross}{\fontsize{25}{38}}
\newcommand{\PraesentationSchriftgroesseGross}{\fontsize{18}{27}}
\newcommand{\PraesentationSchriftgroesseNormal}{\fontsize{14}{21}}
\newcommand{\PraesentationSchriftgroesseKlein}{\fontsize{11}{17}}
\newcommand{\PraesentationSchriftgroesseDreizeiler}{\fontsize{7}{10}}
\newcommand{\PraesentationSchriftgroesseAufzaehlungszeichen}{\fontsize{10}{8}}

\newcommand{\PraesentationAbstandAbsatz}{18pt}
\newcommand{\PraesentationPositionKorrekturOben}{-1cm}
\newcommand{\PraesentationBeispieleSchriftgroessen}{25 | 18 | 14 | 11}

%% Join Operator declarations
\usepackage{amsmath}
\usepackage{amssymb}
\usepackage{ifsym}

\def\ojoin{\setbox0=\hbox{$\bowtie$}%
    \rule[-.02ex]{.25em}{.4pt}\llap{\rule[\ht0]{.25em}{.4pt}}}

\def\leftouterjoin{\mathbin{\ojoin\mkern-5.8mu\bowtie}}
\def\rightouterjoin{\mathbin{\bowtie\mkern-5.8mu\ojoin}}
\def\fullouterjoin{\mathbin{\ojoin\mkern-5.8mu\bowtie\mkern-5.8mu\ojoin}}
\DeclareMathOperator*{\join}{\bowtie}
\DeclareMathOperator*{\leftsemijoin}{\ltimes}
\DeclareMathOperator*{\rightsemijoin}{\rtimes}
\DeclareMathOperator*{\rightantijoin}{\rhd}
\DeclareMathOperator*{\leftantijoin}{\lhd}

\input{../template/res/Praeambel.tex} % Seitenverhältnis 16:9

%%%%%%%%%%%%%%%%%%%%%%%%%%%%%%%%%%%%%%%%%%%%%%%%%%%%%%%%%%%%%%%%%%%%%%%%%%%%%%%%


%%%%%%%%%%%%%%%%%%%%%%%%%%%%%%%%%%%%%%%%%%%%%%%%%%%%%%%%%%%%%%%%%%%%%%%%%%%%%%%%
%%%%%%%%%%%%%%%%%%%%%%%%%%%%%%%%%%%%%%%%%%%%%%%%%%%%%%%%%%%%%%%%%%%%%%%%%%%%%%%%
% TUM-Vorlage: Personenspezifische Informationen
%%%%%%%%%%%%%%%%%%%%%%%%%%%%%%%%%%%%%%%%%%%%%%%%%%%%%%%%%%%%%%%%%%%%%%%%%%%%%%%%
%
% Rechteinhaber:
%     Technische Universität München
%     https://www.tum.de
% 
% Gestaltung:
%     ediundsepp Gestaltungsgesellschaft, München
%     http://www.ediundsepp.de
% 
% Technische Umsetzung:
%     eWorks GmbH, Frankfurt am Main
%     http://www.eworks.de
%
%%%%%%%%%%%%%%%%%%%%%%%%%%%%%%%%%%%%%%%%%%%%%%%%%%%%%%%%%%%%%%%%%%%%%%%%%%%%%%%%

% Für die Person anpassen:

\newcommand{\PersonTitel}{}
\newcommand{\PersonVorname}{Max}
\newcommand{\PersonNachname}{Frühauf}
\newcommand{\PersonStadt}{@Ort@}
\newcommand{\PersonAdresse}{%
    @Adresse@\\%
    @Plz@~\PersonStadt%
}
\newcommand{\PersonTelefon}{@Telefon@}
\newcommand{\PersonFax}{@Fax@}
\newcommand{\PersonEmail}{max.fruehauf@tum.de}
\newcommand{\PersonWebseite}{@Web@}

\newcommand{\FakultaetAnsprechpartner}{@Ansprechpartner@}
% Fakultät:
\newcommand{\FakultaetName}{Fakultät für Informatik}
\newcommand{\LehrstuhlName}{@Lehrstuhlname@}
% Musterdaten:
\newcommand{\EinstellungBankName}{Bayerische Landesbank}
\newcommand{\EinstellungBankIBAN}{DE10700500000000024866}
\newcommand{\EinstellungBankBIC}{BYLADEMM}
\newcommand{\EinstellungSteuernummer}{143/241/80037}
\newcommand{\EinstellungUmsatzsteuerIdentifikationsnummer}{DE811193231}

\hyphenation{} % eigene Silbentrennung                    % !!! DATEI ANPASSEN !!!
%%%%%%%%%%%%%%%%%%%%%%%%%%%%%%%%%%%%%%%%%%%%%%%%%%%%%%%%%%%%%%%%%%%%%%%%%%%%%%%%


% \renewcommand{\PersonTitel}{Dr. rer. nat.}
\newcommand{\Datum}{\today}

\renewcommand{\PraesentationFusszeileZusatz}{| Tutorium Einführung in die Rechnerarchitektur WS 2018/2019}

\title{Tutorübung 1}
\author{\PersonVorname{} \PersonNachname}
\institute[]{\UniversitaetName \\ \FakultaetName}
\date[\Datum]{15. Oktober 2018}


%%%%%%%%%%%%%%%%%%%%%%%%%%%%%%%%%%%%%%%%%%%%%%%%%%%%%%%%%%%%%%%%%%%%%%%%%%%%%%%%
\input{../template/res/Anfang.tex} % !!! NICHT ENTFERNEN !!!
%%%%%%%%%%%%%%%%%%%%%%%%%%%%%%%%%%%%%%%%%%%%%%%%%%%%%%%%%%%%%%%%%%%%%%%%%%%%%%%%
\begin{document}
\setlength{\baselineskip}{\PraesentationAbstandAbsatz}
\setlength{\parskip}{\baselineskip}

%%%%%%%%%%%%%%%%%%%%%%%%%%%%%%%%%%%%%%%%%%%%%%%%%%%%%%%%%%%%%%%%%%%%%%%%%%%%%%%%
% FOLIENSTIL: Standard
% !!!ÄNDERUNG HIER:!!!
\PraesentationMasterStandard

\PraesentationTitelseite % Fügt die Startseite ein

\begin{frame}
    \frametitle{Hausaufgabe 1}
    \vspace{0.5cm}

    Erläutern Sie den Unterschied zwischen dem Relationalen Modell und dem Graphstrukturieren Modell.
    \begin{enumerate}[(a)]
        \item<1-> Nennen Sie ein typisches Einsatzgebiet für das jeweilige Modell.
        \item<2->[] 
        Relationales Modell: 
            \begin{itemize}
                \item Ein CMS.
            \end{itemize}
        Graphstrukturiertes Modell: 
            \begin{itemize}
                \item Datenspeicherung in der Bioinformatik/Medizin.
                \item Speicherung von inhärent graphstrukturierten Daten wie etwa U-Bahn Netze ;-)
            \end{itemize}
    \end{enumerate}
\end{frame}

\begin{frame}
    \frametitle{Hausaufgabe 1}
    \vspace{0.5cm}

    \begin{enumerate}[(b)]
        % \setcounter{enumii}{2}
        \item<1-> Im Datenbankbereich unterscheidet man zwischen Modellen, 
        welche ein festes Schema voraussetzen und anderen, 
        die kein Schema benötigen. Ein Schema ist hierbei eine Vorgabe, 
        wie Daten repräsentiert werden, beispielsweise, dass jede Vorlesung genau 
        eine eindeutige Nummer hat, einen Namen von weniger als 200 Zeichen und eine 
        Semesterwochenstundenzahl.
        \begin{enumerate}[1)]
            \item<1-> Was ist der Vorteil/Nachteil einer solchen Vorgabe für den Anwender?
            \item<2->[] Schema sorgt für sauber und einheitlich abgelegte Daten 
            und bietet Garantien über die Vollständigkeit und das Format der vorhanden Daten. 
            Es geht Flexibilität verloren, da neue Daten, die abgelegt werden müssen, 
            oft eine Schemaänderung notwendig machen.
            \item<1-> Was ist der Vorteil/Nachteil einer solchen Vorgabe für den Entwickler des 
            Datenbanksystems?
            \item<2->[] Ein Schema erlaubt i.A. mehr Optimierungsmöglichkeiten da mehr Informationen 
            und dadurch mehr “Constraints” über die Daten bekannt sind. 
            Das Datenbanksystem kann so Anfragen schneller bearbeiten.
        \end{enumerate}
    \end{enumerate}
\end{frame}

\begin{frame}
    \frametitle{Hausaufgabe 2}
    \vspace{0.5cm}

    Angenommen, lediglich die partielle Funktion
    \[ A \times C \rightarrow B \]
    gilt. 
    Beschriften Sie die Abbildung mit Funktionalitätsangaben.

    Beantworten Sie nun die Frage, wie Funktionalitätsangaben aus partiellen 
    Funktionen und umgekehrt ermittelt werden können. 
    Merken Sie sich die Antwort für die Klausur ;-)
    \pause 

    Eine einfache Daumenregel ist, dass an die Entität, die auf der rechten Seite des 
    Pfeiles einer geltenden partiellen Funktion steht, eine 1 annotiert wird. 
    Es bietet sich daher an, für die sichere Bestimmung der 
    Kardinaliätsangaben grundsätzlich die möglichen partiellen Funktionen aufzustellen und 
    zu überlegen, welche Einschränkungen gewünscht sind.
\end{frame}

\begin{frame}
    \frametitle{Gruppenaufgabe 1}
    \vspace{0.5cm}

    Sie designen eine Webanwendung zur Univerwaltung. 
    Früh entschließen Sie sich zum Einsatz eines Datenbanksystems als 
    Backend für Ihre Daten. 
    Ihr Kollege ist skeptisch und würde die Datenverwaltung lieber selbst implementieren. 
    Überzeugen Sie ihn von Ihrem Entschluss. 
    Finden Sie stichhaltige Antworten auf die folgenden von Ihrem Kollegen in den Raum 
    gestellten Äußerungen:
    \begin{enumerate}[(a)]
        \item Die Installation und Wartung eines Datenbanksystems ist aufwendig, 
        die Erstellung eines eigenen Datenformats ist straight-forward und flexibler.
        \item Mehrbenutzersynchronisation wird in diesem Fall nicht benötigt.
        \item Es ist unsinnig, das jeder Entwickler zunächst eine eigene Anfragesprache (SQL) 
        lernen muss, nur um Daten aus der Datenbank zu extrahieren.
        \item Redundanz ist hilfreich, wieso sollte man auf sie verzichten?
    \end{enumerate}
\end{frame}

\begin{frame}
    \frametitle{Gruppenaufgabe 2}
    \vspace{0.5cm}

    Finden Sie ein Beispiel für ein Problem (bzw. eine Inkonsistenz), 
    die auftreten kann, wenn unkontrolliert parallel auf Daten zugegriffen wird. 
    Ein traditionelles Beispiel hierfür ist eine gegenseitige Bank-Überweisung zwischen zwei 
    Konten A und B. Wenn A einen Betrag \( x \) zu B überweist und B einen Betrag \( x' \) 
    zu A, sollte immer gelten \( Kontostand(A) + Kontostand(B) \) ist konstant, 
    da sonst Geld verschwunden ist. 
    Konstruieren Sie einen Ablauf zweier gegenseitiger Überweisungen, 
    bei dem die Eigenschaft, dass die Kontostandssumme konstant sein soll nach dem 
    Abschluss der zwei Überweisungen verletzt ist.

    \pause
    \begin{table}[]
        \begin{tabular}{l|l|l}
         Schritt & A & B \\ \hline
        1 & \( read(a) \) &  \\
        2 & \( a := a - x \) &  \\
        3 &  & \( read(b) \) \\
        4 &  & \( b := b - x' \) \\
        5 &  & \( read(a) \) \\
        6 &  & \( a := a + x' \) \\
        7 &  & \( write(a) \) \\
        8 & \( write(a) \) &  \\
        \end{tabular}
    \end{table}
\end{frame}
%%%%%%%%%%%%%%%%%%%%%%%%%%%%%%%%%%%%%%%%%%%%%%%%%%%%%%%%%%%%%%%%%%%%%%%%%%%%%%%%
\end{document} % !!! NICHT ENTFERNEN !!!
%%%%%%%%%%%%%%%%%%%%%%%%%%%%%%%%%%%%%%%%%%%%%%%%%%%%%%%%%%%%%%%%%%%%%%%%%%%%%%%%
