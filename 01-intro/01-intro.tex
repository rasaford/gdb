%%%%%%%%%%%%%%%%%%%%%%%%%%%%%%%%%%%%%%%%%%%%%%%%%%%%%%%%%%%%%%%%%%%%%%%%%%%%%%%%
% TUM-Vorlage: Präsentation
%%%%%%%%%%%%%%%%%%%%%%%%%%%%%%%%%%%%%%%%%%%%%%%%%%%%%%%%%%%%%%%%%%%%%%%%%%%%%%%%
%
% Rechteinhaber:
%     Technische Universität München
%     https://www.tum.de
% 
% Gestaltung:
%     ediundsepp Gestaltungsgesellschaft, München
%     http://www.ediundsepp.de
% 
% Technische Umsetzung:
%     eWorks GmbH, Frankfurt am Main
%     http://www.eworks.de
%
%%%%%%%%%%%%%%%%%%%%%%%%%%%%%%%%%%%%%%%%%%%%%%%%%%%%%%%%%%%%%%%%%%%%%%%%%%%%%%%%


%%%%%%%%%%%%%%%%%%%%%%%%%%%%%%%%%%%%%%%%%%%%%%%%%%%%%%%%%%%%%%%%%%%%%%%%%%%%%%%%
% Zur Wahl des Seitenverhältnisses bitte einen der beiden folgenden Befehle
% auskommentieren und den ausführen lassen:
%\documentclass[aspectratio=169]{beamer}
\documentclass[t,aspectratio=169]{beamer}
\usepackage[
	orientation=landscape,
	size=custom,
	width=25.4,
	height=14.2875,
	scale=0.5
]{beamerposter}
\usepackage{verbatim}
\usepackage{booktabs}
\usepackage{verbatim}
\usepackage[cache=false]{minted}
\usemintedstyle{tango}
\definecolor{codeBG}{rgb}{0.97, 0.97, 0.99}
\setminted{linenos=true, bgcolor=codeBG, mathescape=true}

\newcommand{\PraesentationSchriftgroesseSehrGross}{\fontsize{25}{38}}
\newcommand{\PraesentationSchriftgroesseGross}{\fontsize{18}{27}}
\newcommand{\PraesentationSchriftgroesseNormal}{\fontsize{14}{21}}
\newcommand{\PraesentationSchriftgroesseKlein}{\fontsize{11}{17}}
\newcommand{\PraesentationSchriftgroesseDreizeiler}{\fontsize{7}{10}}
\newcommand{\PraesentationSchriftgroesseAufzaehlungszeichen}{\fontsize{10}{8}}

\newcommand{\PraesentationAbstandAbsatz}{18pt}
\newcommand{\PraesentationPositionKorrekturOben}{-1cm}
\newcommand{\PraesentationBeispieleSchriftgroessen}{25 | 18 | 14 | 11}

%% Join Operator declarations
\usepackage{amsmath}
\usepackage{amssymb}
\usepackage{ifsym}

\def\ojoin{\setbox0=\hbox{$\bowtie$}%
    \rule[-.02ex]{.25em}{.4pt}\llap{\rule[\ht0]{.25em}{.4pt}}}

\def\leftouterjoin{\mathbin{\ojoin\mkern-5.8mu\bowtie}}
\def\rightouterjoin{\mathbin{\bowtie\mkern-5.8mu\ojoin}}
\def\fullouterjoin{\mathbin{\ojoin\mkern-5.8mu\bowtie\mkern-5.8mu\ojoin}}
\DeclareMathOperator*{\join}{\bowtie}
\DeclareMathOperator*{\leftsemijoin}{\ltimes}
\DeclareMathOperator*{\rightsemijoin}{\rtimes}
\DeclareMathOperator*{\rightantijoin}{\rhd}
\DeclareMathOperator*{\leftantijoin}{\lhd}

\input{../template/res/Praeambel.tex} % Seitenverhältnis 16:9

%%%%%%%%%%%%%%%%%%%%%%%%%%%%%%%%%%%%%%%%%%%%%%%%%%%%%%%%%%%%%%%%%%%%%%%%%%%%%%%%


%%%%%%%%%%%%%%%%%%%%%%%%%%%%%%%%%%%%%%%%%%%%%%%%%%%%%%%%%%%%%%%%%%%%%%%%%%%%%%%%
%%%%%%%%%%%%%%%%%%%%%%%%%%%%%%%%%%%%%%%%%%%%%%%%%%%%%%%%%%%%%%%%%%%%%%%%%%%%%%%%
% TUM-Vorlage: Personenspezifische Informationen
%%%%%%%%%%%%%%%%%%%%%%%%%%%%%%%%%%%%%%%%%%%%%%%%%%%%%%%%%%%%%%%%%%%%%%%%%%%%%%%%
%
% Rechteinhaber:
%     Technische Universität München
%     https://www.tum.de
% 
% Gestaltung:
%     ediundsepp Gestaltungsgesellschaft, München
%     http://www.ediundsepp.de
% 
% Technische Umsetzung:
%     eWorks GmbH, Frankfurt am Main
%     http://www.eworks.de
%
%%%%%%%%%%%%%%%%%%%%%%%%%%%%%%%%%%%%%%%%%%%%%%%%%%%%%%%%%%%%%%%%%%%%%%%%%%%%%%%%

% Für die Person anpassen:

\newcommand{\PersonTitel}{}
\newcommand{\PersonVorname}{Max}
\newcommand{\PersonNachname}{Frühauf}
\newcommand{\PersonStadt}{@Ort@}
\newcommand{\PersonAdresse}{%
    @Adresse@\\%
    @Plz@~\PersonStadt%
}
\newcommand{\PersonTelefon}{@Telefon@}
\newcommand{\PersonFax}{@Fax@}
\newcommand{\PersonEmail}{max.fruehauf@tum.de}
\newcommand{\PersonWebseite}{@Web@}

\newcommand{\FakultaetAnsprechpartner}{@Ansprechpartner@}
% Fakultät:
\newcommand{\FakultaetName}{Fakultät für Informatik}
\newcommand{\LehrstuhlName}{@Lehrstuhlname@}
% Musterdaten:
\newcommand{\EinstellungBankName}{Bayerische Landesbank}
\newcommand{\EinstellungBankIBAN}{DE10700500000000024866}
\newcommand{\EinstellungBankBIC}{BYLADEMM}
\newcommand{\EinstellungSteuernummer}{143/241/80037}
\newcommand{\EinstellungUmsatzsteuerIdentifikationsnummer}{DE811193231}

\hyphenation{} % eigene Silbentrennung                    % !!! DATEI ANPASSEN !!!
%%%%%%%%%%%%%%%%%%%%%%%%%%%%%%%%%%%%%%%%%%%%%%%%%%%%%%%%%%%%%%%%%%%%%%%%%%%%%%%%


% \renewcommand{\PersonTitel}{Dr. rer. nat.}
\newcommand{\Datum}{\today}

\renewcommand{\PraesentationFusszeileZusatz}{| Tutorium Grundlagen: Datenbanken WS 18/19}

\title{Tutorübung 1}
\author{\PersonVorname{} \PersonNachname}
\institute[]{\UniversitaetName \\ \FakultaetName}
\date[\Datum]{15. Oktober 2018}


%%%%%%%%%%%%%%%%%%%%%%%%%%%%%%%%%%%%%%%%%%%%%%%%%%%%%%%%%%%%%%%%%%%%%%%%%%%%%%%%
\input{../template/res/Anfang.tex} % !!! NICHT ENTFERNEN !!!
%%%%%%%%%%%%%%%%%%%%%%%%%%%%%%%%%%%%%%%%%%%%%%%%%%%%%%%%%%%%%%%%%%%%%%%%%%%%%%%%
\begin{document}
\setlength{\baselineskip}{\PraesentationAbstandAbsatz}
\setlength{\parskip}{\baselineskip}

%%%%%%%%%%%%%%%%%%%%%%%%%%%%%%%%%%%%%%%%%%%%%%%%%%%%%%%%%%%%%%%%%%%%%%%%%%%%%%%%
% FOLIENSTIL: Standard
% !!!ÄNDERUNG HIER:!!!
\PraesentationMasterStandard

\PraesentationTitelseite % Fügt die Startseite ein

\begin{frame}
	\frametitle{Hausaufgabe 1}
	\vspace{0.25cm}

	\begin{multicols}{2}
		Als moderner Netzbürger besitzen Sie einen Twitter-Account mit 150 Followern. Nach der Vorlesung schreiben Sie
		folgenden Tweet:

		\begin{enumerate}[a)]
			\only<1-2> {
			\item Welche Daten werden durch das Absenden des Tweets generiert? \\
			      Wo werden diese im System von Twitter gespeichert? (Auch an Metadaten denken) \\
			      Welche Daten werden generiert, wenn der Tweet einen Like erhält?
			      }
			      \setcounter{enumi}{1}
			\item<3-> Durch die DSGVO müssen u.a. personenbezogene Daten auf Anfrage gelöscht werden.
			      Welche Daten müssen bei einer solchen Anfrage gelöscht werden?
			      Welche Probleme können dabei auftreten?
		\end{enumerate}
		\vfill\columnbreak
		\begin{center}
			\includegraphics[height=.2\paperheight]{./tweet.png}
		\end{center}
		\vspace{-1cm}

		\begin{enumerate}[a)]
			\only<2> {
			\item \begin{itemize}
				      \color{TUMBlau}
				      \item Zuordnung zum Account @netzbürger42, Datum, Benachrichtigung für @DLR\_de,
				            Metadaten des Hashtags, Benutzerdaten.
				      \item Tweets und Nutzerdaten werden wahrscheinlich getrennt gespeichert.
				      \item Beim Like eine Verbindung zwischen beiden Accounts gespeichert.
			      \end{itemize}
			      }
			      \setcounter{enumi}{1}
			\item <4-> \begin{itemize}
				      \color{TUMBlau}
				      \item Tweets, Benutzerkonto (Accountname, Passwort-Hashes, Geburtstag ...).
				      \item Problem bei der Interaktion mit anderen,
				            da die Konsistenz der Daten garantiert werden muss. \\
				            Löschen aus Backups kann schwierig sein.
			      \end{itemize}
		\end{enumerate}
	\end{multicols}
\end{frame}


\begin{frame}
	\frametitle{Hausaufgabe 2}
	\vspace{0.25cm}

	\begin{multicols}{2}
		\only<1-4> {
			Ignorieren Sie die Funktionalitätsangaben und beantworten Sie:
			\begin{enumerate}[a)]
				\item<1-> Wie viele partielle Funktionen der Form \( A \times B \rightarrow C \)
				      können in einer ternären Beziehung auftreten? \\
				      (Ignorieren Sie die Reihenfolge auf der linken Seite)
				\item<1-> Nennen Sie alle möglichen partiellen Beziehungen in der hier
				      gezeigten Beziehung "nimmt teil".
				\item<3-> Nennen Sie für jede Funktion in Prosa,
				      welche Einschränkung diese darstellt, falls sie gilt.
			\end{enumerate}
		}
		% \only<
		Unter Berücksichtigung der Funktionalitätsangaben:
		\begin{enumerate}[a)]
			\item Welche partiellen Funktionen gelten hier?
		\end{enumerate}
		\vfill\columnbreak

		\begin{center}
			\includegraphics[height=.2\paperheight]{./er.png}
		\end{center}
		\vspace{-1cm}
		\begin{enumerate}[a)]
			\color{TUMBlau}
			\item<2-> Es gibt 3 mögliche partielle Funktionen.
			\item<2-> \begin{enumerate}[1.]
				      \color{TUMBlau}
				      \item \( Tutor \times Uebung \rightarrow Student \)
				      \item \( Student \times Tutor \rightarrow Uebung \)
				      \item \( Student \times Uebung \rightarrow Tutor \)
			      \end{enumerate}
			\item<4->
			      Ein Tutor darf pro Übung nur einen Studenten haben. \\
			      Ein Student darf bei einem Tutor nur eine Übung besuchen. \\
			      Ein Student darf in einer Übung nur einen Tutor haben. \\
		\end{enumerate}
	\end{multicols}
\end{frame}

% \begin{frame}
% 	\frametitle{Hausaufgabe 2}
% 	\vspace{0.25cm}

% 	\begin{multicols}{2}
% 		% Ignorieren Sie die Funktionalitätsangaben und beantworten Sie:
% 		% \begin{enumerate}[a)]
% 		% 	\item<1-> Wie viele partielle Funktionen der Form \( A \times B \rightarrow C \)
% 		% 	      können in einer ternären Beziehung auftreten? \\
% 		% 	      (Ignorieren Sie die Reihenfolge auf der linken Seite)
% 		% 	\item<1-> Nennen Sie alle möglichen partiellen Beziehungen in der hier
% 		% 	      gezeigten Beziehung "nimmt teil".
% 		% 	\item<3-> Nennen Sie für jede Funktion in Prosa,
% 		% 	      welche Einschränkung diese darstellt, falls sie gilt.
% 		% \end{enumerate}
% 		Unter Berücksichtigung der Funktionalitätsangaben:
% 		\begin{enumerate}[a)]
% 			\setcounter{enumi}{3}
% 			\item Welche partiellen Funktionen gelten hier?
% 		\end{enumerate}

% 		\begin{enumerate}[a)]
% 			\setcounter{enumi}{3}
% 			\color{TUMBlau}
% 			\item Funktionen 2. und 3. passen, da diese eine 1 auf der rechten Seite haben und
% 			      im Übungsbetrieb der Fall sind.
% 		\end{enumerate}
% 		\vfill\columnbreak

% 		\begin{center}
% 			\includegraphics[height=.2\paperheight]{./er.png}
% 		\end{center}
% 		\vspace{-1cm}
% 		% \begin{enumerate}[a)]
% 		% 	\color{TUMBlau}
% 		% 	\item<2-> Es gibt 3 mögliche partielle Funktionen.
% 		% 	\item<2-> \begin{enumerate}[1.]
% 		% 		      \color{TUMBlau}
% 		% 		      \item \( Tutor \times Uebung \rightarrow Student \)
% 		% 		      \item \( Student \times Tutor \rightarrow Uebung \)
% 		% 		      \item \( Student \times Uebung \rightarrow Tutor \)
% 		% 	      \end{enumerate}
% 		% 	\item<4->
% 		% 	      Ein Tutor darf pro Übung nur einen Studenten haben. \\
% 		% 	      Ein Student darf bei einem Tutor nur eine Übung besuchen. \\
% 		% 	      Ein Student darf in einer Übung nur einen Tutor haben. \\
% 		% \end{enumerate}
% 	\end{multicols}
% \end{frame}

\begin{frame}
	\frametitle{Gruppenaufgabe 1}
	\vspace{0.25cm}

	Sie designen eine Webanwendung zur Univerwaltung.
	Früh entschließen Sie sich zum Einsatz eines Datenbanksystems als
	Backend für Ihre Daten.
	Ihr Kollege ist skeptisch und würde die Datenverwaltung lieber selbst implementieren.
	Überzeugen Sie ihn von Ihrem Entschluss.
	Finden Sie stichhaltige Antworten auf die folgenden von Ihrem Kollegen in den Raum
	gestellten Äußerungen:
	\begin{enumerate}[a)]
		\item Die Installation und Wartung eines Datenbanksystems ist aufwendig,
		      die Erstellung eines eigenen Datenformats ist straight-forward und flexibler.
		\item Mehrbenutzersynchronisation wird in diesem Fall nicht benötigt.
		\item Es ist unsinnig, das jeder Entwickler zunächst eine eigene Anfragesprache (SQL)
		      lernen muss, nur um Daten aus der Datenbank zu extrahieren.
		\item Redundanz ist hilfreich, wieso sollte man auf sie verzichten?
	\end{enumerate}
\end{frame}

\begin{frame}
	\frametitle{Gruppenaufgabe 2}
	\vspace{0.25cm}

	Finden Sie ein Beispiel für ein Problem (bzw. eine Inkonsistenz),
	die auftreten kann, wenn unkontrolliert parallel auf Daten zugegriffen wird.
	Ein traditionelles Beispiel hierfür ist eine gegenseitige Bank-Überweisung zwischen zwei
	Konten A und B. Wenn A einen Betrag \( x \) zu B überweist und B einen Betrag \( x' \)
	zu A, sollte immer gelten \( Kontostand(A) + Kontostand(B) \) ist konstant,
	da sonst Geld verschwunden ist.
	Konstruieren Sie einen Ablauf zweier gegenseitiger Überweisungen,
	bei dem die Eigenschaft, dass die Kontostandssumme konstant sein soll nach dem
	Abschluss der zwei Überweisungen verletzt ist.

	\pause
	\begin{table}[]
		\color{TUMBlau}
		\begin{tabular}{c|c|c}
			Schritt & A                & B                 \\ \hline
			1       & \( read(a) \)    &                   \\
			2       & \( a := a - x \) &                   \\
			3       &                  & \( read(b) \)     \\
			4       &                  & \( b := b - x' \) \\
			5       &                  & \( read(a) \)     \\
			6       &                  & \( a := a + x' \) \\
			7       &                  & \( write(a) \)    \\
			8       & \( write(a) \)   &                   \\
		\end{tabular}
	\end{table}

\end{frame}
%%%%%%%%%%%%%%%%%%%%%%%%%%%%%%%%%%%%%%%%%%%%%%%%%%%%%%%%%%%%%%%%%%%%%%%%%%%%%%%%
\end{document} % !!! NICHT ENTFERNEN !!!
%%%%%%%%%%%%%%%%%%%%%%%%%%%%%%%%%%%%%%%%%%%%%%%%%%%%%%%%%%%%%%%%%%%%%%%%%%%%%%%%
