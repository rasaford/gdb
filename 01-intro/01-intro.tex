%%%%%%%%%%%%%%%%%%%%%%%%%%%%%%%%%%%%%%%%%%%%%%%%%%%%%%%%%%%%%%%%%%%%%%%%%%%%%%%%
% TUM-Vorlage: Präsentation
%%%%%%%%%%%%%%%%%%%%%%%%%%%%%%%%%%%%%%%%%%%%%%%%%%%%%%%%%%%%%%%%%%%%%%%%%%%%%%%%
%
% Rechteinhaber:
%     Technische Universität München
%     https://www.tum.de
% 
% Gestaltung:
%     ediundsepp Gestaltungsgesellschaft, München
%     http://www.ediundsepp.de
% 
% Technische Umsetzung:
%     eWorks GmbH, Frankfurt am Main
%     http://www.eworks.de
%
%%%%%%%%%%%%%%%%%%%%%%%%%%%%%%%%%%%%%%%%%%%%%%%%%%%%%%%%%%%%%%%%%%%%%%%%%%%%%%%%


%%%%%%%%%%%%%%%%%%%%%%%%%%%%%%%%%%%%%%%%%%%%%%%%%%%%%%%%%%%%%%%%%%%%%%%%%%%%%%%%
% Zur Wahl des Seitenverhältnisses bitte einen der beiden folgenden Befehle
% auskommentieren und den ausführen lassen:
%\documentclass[aspectratio=169]{beamer}
\documentclass[t,aspectratio=169]{beamer}
\usepackage[
	orientation=landscape,
	size=custom,
	width=25.4,
	height=14.2875,
	scale=0.5
]{beamerposter}
\usepackage{verbatim}
\usepackage{booktabs}
\usepackage{verbatim}
\usepackage[cache=false]{minted}
\usemintedstyle{tango}
\definecolor{codeBG}{rgb}{0.97, 0.97, 0.99}
\setminted{linenos=true, bgcolor=codeBG, mathescape=true}

\newcommand{\PraesentationSchriftgroesseSehrGross}{\fontsize{25}{38}}
\newcommand{\PraesentationSchriftgroesseGross}{\fontsize{18}{27}}
\newcommand{\PraesentationSchriftgroesseNormal}{\fontsize{14}{21}}
\newcommand{\PraesentationSchriftgroesseKlein}{\fontsize{11}{17}}
\newcommand{\PraesentationSchriftgroesseDreizeiler}{\fontsize{7}{10}}
\newcommand{\PraesentationSchriftgroesseAufzaehlungszeichen}{\fontsize{10}{8}}

\newcommand{\PraesentationAbstandAbsatz}{18pt}
\newcommand{\PraesentationPositionKorrekturOben}{-1cm}
\newcommand{\PraesentationBeispieleSchriftgroessen}{25 | 18 | 14 | 11}

%% Join Operator declarations
\usepackage{amsmath}
\usepackage{amssymb}
\usepackage{ifsym}

\def\ojoin{\setbox0=\hbox{$\bowtie$}%
    \rule[-.02ex]{.25em}{.4pt}\llap{\rule[\ht0]{.25em}{.4pt}}}

\def\leftouterjoin{\mathbin{\ojoin\mkern-5.8mu\bowtie}}
\def\rightouterjoin{\mathbin{\bowtie\mkern-5.8mu\ojoin}}
\def\fullouterjoin{\mathbin{\ojoin\mkern-5.8mu\bowtie\mkern-5.8mu\ojoin}}
\DeclareMathOperator*{\join}{\bowtie}
\DeclareMathOperator*{\leftsemijoin}{\ltimes}
\DeclareMathOperator*{\rightsemijoin}{\rtimes}
\DeclareMathOperator*{\rightantijoin}{\rhd}
\DeclareMathOperator*{\leftantijoin}{\lhd}

\input{../template/res/Praeambel.tex} % Seitenverhältnis 16:9

%%%%%%%%%%%%%%%%%%%%%%%%%%%%%%%%%%%%%%%%%%%%%%%%%%%%%%%%%%%%%%%%%%%%%%%%%%%%%%%%


%%%%%%%%%%%%%%%%%%%%%%%%%%%%%%%%%%%%%%%%%%%%%%%%%%%%%%%%%%%%%%%%%%%%%%%%%%%%%%%%
%%%%%%%%%%%%%%%%%%%%%%%%%%%%%%%%%%%%%%%%%%%%%%%%%%%%%%%%%%%%%%%%%%%%%%%%%%%%%%%%
% TUM-Vorlage: Personenspezifische Informationen
%%%%%%%%%%%%%%%%%%%%%%%%%%%%%%%%%%%%%%%%%%%%%%%%%%%%%%%%%%%%%%%%%%%%%%%%%%%%%%%%
%
% Rechteinhaber:
%     Technische Universität München
%     https://www.tum.de
% 
% Gestaltung:
%     ediundsepp Gestaltungsgesellschaft, München
%     http://www.ediundsepp.de
% 
% Technische Umsetzung:
%     eWorks GmbH, Frankfurt am Main
%     http://www.eworks.de
%
%%%%%%%%%%%%%%%%%%%%%%%%%%%%%%%%%%%%%%%%%%%%%%%%%%%%%%%%%%%%%%%%%%%%%%%%%%%%%%%%

% Für die Person anpassen:

\newcommand{\PersonTitel}{}
\newcommand{\PersonVorname}{Max}
\newcommand{\PersonNachname}{Frühauf}
\newcommand{\PersonStadt}{@Ort@}
\newcommand{\PersonAdresse}{%
    @Adresse@\\%
    @Plz@~\PersonStadt%
}
\newcommand{\PersonTelefon}{@Telefon@}
\newcommand{\PersonFax}{@Fax@}
\newcommand{\PersonEmail}{max.fruehauf@tum.de}
\newcommand{\PersonWebseite}{@Web@}

\newcommand{\FakultaetAnsprechpartner}{@Ansprechpartner@}
% Fakultät:
\newcommand{\FakultaetName}{Fakultät für Informatik}
\newcommand{\LehrstuhlName}{@Lehrstuhlname@}
% Musterdaten:
\newcommand{\EinstellungBankName}{Bayerische Landesbank}
\newcommand{\EinstellungBankIBAN}{DE10700500000000024866}
\newcommand{\EinstellungBankBIC}{BYLADEMM}
\newcommand{\EinstellungSteuernummer}{143/241/80037}
\newcommand{\EinstellungUmsatzsteuerIdentifikationsnummer}{DE811193231}

\hyphenation{} % eigene Silbentrennung                    % !!! DATEI ANPASSEN !!!
%%%%%%%%%%%%%%%%%%%%%%%%%%%%%%%%%%%%%%%%%%%%%%%%%%%%%%%%%%%%%%%%%%%%%%%%%%%%%%%%


% \renewcommand{\PersonTitel}{Dr. rer. nat.}
\newcommand{\Datum}{\today}

\renewcommand{\PraesentationFusszeileZusatz}{| Tutorium Einführung in die Rechnerarchitektur WS 2018/2019}

\title{Tutorübung 1}
\author{\PersonVorname{} \PersonNachname}
\institute[]{\UniversitaetName \\ \FakultaetName}
\date[\Datum]{15. Oktober 2018}


%%%%%%%%%%%%%%%%%%%%%%%%%%%%%%%%%%%%%%%%%%%%%%%%%%%%%%%%%%%%%%%%%%%%%%%%%%%%%%%%
\input{../template/res/Anfang.tex} % !!! NICHT ENTFERNEN !!!
%%%%%%%%%%%%%%%%%%%%%%%%%%%%%%%%%%%%%%%%%%%%%%%%%%%%%%%%%%%%%%%%%%%%%%%%%%%%%%%%
\begin{document}
\setlength{\baselineskip}{\PraesentationAbstandAbsatz}
\setlength{\parskip}{\baselineskip}

%%%%%%%%%%%%%%%%%%%%%%%%%%%%%%%%%%%%%%%%%%%%%%%%%%%%%%%%%%%%%%%%%%%%%%%%%%%%%%%%
% FOLIENSTIL: Standard
% !!!ÄNDERUNG HIER:!!!
\PraesentationMasterStandard

\PraesentationTitelseite % Fügt die Startseite ein

\begin{frame}
    \frametitle{Zahlenbasen}
    \vspace{0.5cm}

    \begin{multicols}{2}
        \begin{enumerate}
            \item Zweierpotenzen bis \( 2^{64} \) 
            \item Konvertierung Dezimal \( \rightarrow \) Binär: \\
                \( 42_{10} \) \\
                \( 255_{10} \) \\
                \( 4369_{10} \)
            \item Konvertierung Binär \( \rightarrow \) Dezimal: \\
                \( 110110_{2} \) \\
                \( 10010110_{2} \) \\
                \( 110011001100_{2} \) \\
            \item Konvertierung Dezimal \( \rightarrow \) Hexadezimal: \\
                \( 42_{10} \) \\
                \( 255_{10} \) \\
                \( 4269_{10} \) \\
        \end{enumerate}
        \vfill\columnbreak
        \pause

        \begin{enumerate}
            \item Lösung: Tafel
            \item Lösung: \\
                \( = 101010_{2} \) \\
                \( = 11111111_{2} \) \\
                \( = 1000100010001_{2} \)
            \item Lösung: \\
                \( = 54_{10} \) \\
                \( = 150_{10} \) \\
                \( = 3276_{10} \) \\
            \item Lösung: \\
                \( = 2A_{16} \) \\
                \( = FF_{16} \) \\
                \( = 1111_{16} \) \\
        \end{enumerate}
    \end{multicols}
\end{frame}

\begin{frame}
    \frametitle{Zahlenbasen 2}  
    \vspace{0.5cm}

    \begin{multicols}{2}
        \begin{enumerate}
            \setcounter{enumi}{4}
            \item Konvertierung Binär \( \rightarrow \) Hexadezimal: \\
                \( = 101010_{2} \) \\
                \( = 11111111_{2} \) \\
                \( = 1000100010001_{2} \)
            \item Konvertierung Hexadezimal \( \rightarrow \) Dezimal: \\
                \( BEEF_{16} \) \\
                \( DEAD_{16} \) \\
                \( FEED_{16} \) \\
            \item Bensonderheiten der folgengen Hexwerte. Wofür kann man sie verwenden? \\
                \( 0xFF \) \\
                \( 0x00 \) \\
                \( 0x55 \) \\
                \( 0xAA \) \\
        \end{enumerate}
        \vfill\columnbreak
        \pause

        \begin{enumerate}
            \setcounter{enumi}{4}
            \item Lösung: \\
                \( = 2A_{16} \) \\
                \( = FF_{16} \) \\
                \( = 1111_{16} \)
            \item Lösung: \\
                \( = 28879_{10} \) \\
                \( = 57005_{10} \) \\
                \( = 65261_{10} \) \\
            \item Lösung: \\
                \begin{itemize}
                    \item Nutzung für Speichertests
                    \item Stress von seriellen / parallelen Übertragungsystemen \( ( 0xFF_{16}, 0x00_{16} ) \)
                    \item Test auf Übersprechen zwischen zwei Bitleitungen \( ( 0x55_{16}, 0xAA_{16} ) \)
                \end{itemize}
        \end{enumerate}
    \end{multicols}
\end{frame}

\begin{frame}
    \frametitle{Rechenoperationen}
    \vspace{0.5cm}

    \begin{multicols}{2}
        \begin{enumerate}
            \setcounter{enumi}{7}
            \item Addition in Binär \& Dezimal dann Ergebnisse vergleichen \\
                \( 011001100_2 + 010110011_2 \) \\
                \( 0001111011_2 + 0101011001_2 \) \\
            \item Subtraktion in Binär \& Dezimal dann Ergebnisse vergleichen \\
                \( 011001100_2 - 010110011_2 \) \\
                \( 0001111011_2 - 0101011001_2 \) \\
        \end{enumerate}
        \vfill\columnbreak
        \pause

        \begin{enumerate}
            \setcounter{enumi}{7}
            \item Lösung: \\
                \( 011001100_2 + 010110011_2 = 0101111111_2 \) \\
                \( 0001111011_2 + 0101011001_2 = 0111010100_2 \) \\
            \item Lösung: \\
                \( 011001100_2 - 010110011_2 = 000011001_2 \) \\
                \( 0001111011_2 - 0101011001_2 =  (\hdots 11)1100100010_2\) \\
        \end{enumerate}
    \end{multicols}
\end{frame}

%%%%%%%%%%%%%%%%%%%%%%%%%%%%%%%%%%%%%%%%%%%%%%%%%%%%%%%%%%%%%%%%%%%%%%%%%%%%%%%%
\end{document} % !!! NICHT ENTFERNEN !!!
%%%%%%%%%%%%%%%%%%%%%%%%%%%%%%%%%%%%%%%%%%%%%%%%%%%%%%%%%%%%%%%%%%%%%%%%%%%%%%%%
