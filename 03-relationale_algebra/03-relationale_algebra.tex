%%%%%%%%%%%%%%%%%%%%%%%%%%%%%%%%%%%%%%%%%%%%%%%%%%%%%%%%%%%%%%%%%%%%%%%%%%%%%%%%
% TUM-Vorlage: Präsentation
%%%%%%%%%%%%%%%%%%%%%%%%%%%%%%%%%%%%%%%%%%%%%%%%%%%%%%%%%%%%%%%%%%%%%%%%%%%%%%%%
%
% Rechteinhaber:
%     Technische Universität München
%     https://www.tum.de
% 
% Gestaltung:
%     ediundsepp Gestaltungsgesellschaft, München
%     http://www.ediundsepp.de
% 
% Technische Umsetzung:
%     eWorks GmbH, Frankfurt am Main
%     http://www.eworks.de
%
%%%%%%%%%%%%%%%%%%%%%%%%%%%%%%%%%%%%%%%%%%%%%%%%%%%%%%%%%%%%%%%%%%%%%%%%%%%%%%%%


%%%%%%%%%%%%%%%%%%%%%%%%%%%%%%%%%%%%%%%%%%%%%%%%%%%%%%%%%%%%%%%%%%%%%%%%%%%%%%%%
% Zur Wahl des Seitenverhältnisses bitte einen der beiden folgenden Befehle
% auskommentieren und den ausführen lassen:
%\documentclass[aspectratio=169]{beamer}
\documentclass[t,aspectratio=169]{beamer}
\usepackage[
	orientation=landscape,
	size=custom,
	width=25.4,
	height=14.2875,
	scale=0.5
]{beamerposter}
\usepackage{verbatim}
\usepackage{booktabs}
\usepackage{verbatim}
\usepackage[cache=false]{minted}
\usemintedstyle{tango}
\definecolor{codeBG}{rgb}{0.97, 0.97, 0.99}
\setminted{linenos=true, bgcolor=codeBG, mathescape=true}

\newcommand{\PraesentationSchriftgroesseSehrGross}{\fontsize{25}{38}}
\newcommand{\PraesentationSchriftgroesseGross}{\fontsize{18}{27}}
\newcommand{\PraesentationSchriftgroesseNormal}{\fontsize{14}{21}}
\newcommand{\PraesentationSchriftgroesseKlein}{\fontsize{11}{17}}
\newcommand{\PraesentationSchriftgroesseDreizeiler}{\fontsize{7}{10}}
\newcommand{\PraesentationSchriftgroesseAufzaehlungszeichen}{\fontsize{10}{8}}

\newcommand{\PraesentationAbstandAbsatz}{18pt}
\newcommand{\PraesentationPositionKorrekturOben}{-1cm}
\newcommand{\PraesentationBeispieleSchriftgroessen}{25 | 18 | 14 | 11}

%% Join Operator declarations
\usepackage{amsmath}
\usepackage{amssymb}
\usepackage{ifsym}

\def\ojoin{\setbox0=\hbox{$\bowtie$}%
    \rule[-.02ex]{.25em}{.4pt}\llap{\rule[\ht0]{.25em}{.4pt}}}

\def\leftouterjoin{\mathbin{\ojoin\mkern-5.8mu\bowtie}}
\def\rightouterjoin{\mathbin{\bowtie\mkern-5.8mu\ojoin}}
\def\fullouterjoin{\mathbin{\ojoin\mkern-5.8mu\bowtie\mkern-5.8mu\ojoin}}
\DeclareMathOperator*{\join}{\bowtie}
\DeclareMathOperator*{\leftsemijoin}{\ltimes}
\DeclareMathOperator*{\rightsemijoin}{\rtimes}
\DeclareMathOperator*{\rightantijoin}{\rhd}
\DeclareMathOperator*{\leftantijoin}{\lhd}

\input{../template/res/Praeambel.tex} % Seitenverhältnis 16:9

%%%%%%%%%%%%%%%%%%%%%%%%%%%%%%%%%%%%%%%%%%%%%%%%%%%%%%%%%%%%%%%%%%%%%%%%%%%%%%%%


%%%%%%%%%%%%%%%%%%%%%%%%%%%%%%%%%%%%%%%%%%%%%%%%%%%%%%%%%%%%%%%%%%%%%%%%%%%%%%%%
%%%%%%%%%%%%%%%%%%%%%%%%%%%%%%%%%%%%%%%%%%%%%%%%%%%%%%%%%%%%%%%%%%%%%%%%%%%%%%%%
% TUM-Vorlage: Personenspezifische Informationen
%%%%%%%%%%%%%%%%%%%%%%%%%%%%%%%%%%%%%%%%%%%%%%%%%%%%%%%%%%%%%%%%%%%%%%%%%%%%%%%%
%
% Rechteinhaber:
%     Technische Universität München
%     https://www.tum.de
% 
% Gestaltung:
%     ediundsepp Gestaltungsgesellschaft, München
%     http://www.ediundsepp.de
% 
% Technische Umsetzung:
%     eWorks GmbH, Frankfurt am Main
%     http://www.eworks.de
%
%%%%%%%%%%%%%%%%%%%%%%%%%%%%%%%%%%%%%%%%%%%%%%%%%%%%%%%%%%%%%%%%%%%%%%%%%%%%%%%%

% Für die Person anpassen:

\newcommand{\PersonTitel}{}
\newcommand{\PersonVorname}{Max}
\newcommand{\PersonNachname}{Frühauf}
\newcommand{\PersonStadt}{@Ort@}
\newcommand{\PersonAdresse}{%
    @Adresse@\\%
    @Plz@~\PersonStadt%
}
\newcommand{\PersonTelefon}{@Telefon@}
\newcommand{\PersonFax}{@Fax@}
\newcommand{\PersonEmail}{max.fruehauf@tum.de}
\newcommand{\PersonWebseite}{@Web@}

\newcommand{\FakultaetAnsprechpartner}{@Ansprechpartner@}
% Fakultät:
\newcommand{\FakultaetName}{Fakultät für Informatik}
\newcommand{\LehrstuhlName}{@Lehrstuhlname@}
% Musterdaten:
\newcommand{\EinstellungBankName}{Bayerische Landesbank}
\newcommand{\EinstellungBankIBAN}{DE10700500000000024866}
\newcommand{\EinstellungBankBIC}{BYLADEMM}
\newcommand{\EinstellungSteuernummer}{143/241/80037}
\newcommand{\EinstellungUmsatzsteuerIdentifikationsnummer}{DE811193231}

\hyphenation{} % eigene Silbentrennung                    % !!! DATEI ANPASSEN !!!
%%%%%%%%%%%%%%%%%%%%%%%%%%%%%%%%%%%%%%%%%%%%%%%%%%%%%%%%%%%%%%%%%%%%%%%%%%%%%%%%


% \renewcommand{\PersonTitel}{Dr. rer. nat.}
\newcommand{\Datum}{\today}

\renewcommand{\PraesentationFusszeileZusatz}{| Tutorium Grundlagen: Datenbanken WS 18/19}

\title{Tutorübung 3}
\author{\PersonVorname{} \PersonNachname}
\institute[]{\UniversitaetName \\ \FakultaetName}
\date[\Datum]{15. Oktober 2018}


%%%%%%%%%%%%%%%%%%%%%%%%%%%%%%%%%%%%%%%%%%%%%%%%%%%%%%%%%%%%%%%%%%%%%%%%%%%%%%%%
\input{../template/res/Anfang.tex} % !!! NICHT ENTFERNEN !!!
%%%%%%%%%%%%%%%%%%%%%%%%%%%%%%%%%%%%%%%%%%%%%%%%%%%%%%%%%%%%%%%%%%%%%%%%%%%%%%%%
\begin{document}
\setlength{\baselineskip}{\PraesentationAbstandAbsatz}
\setlength{\parskip}{\baselineskip}

%%%%%%%%%%%%%%%%%%%%%%%%%%%%%%%%%%%%%%%%%%%%%%%%%%%%%%%%%%%%%%%%%%%%%%%%%%%%%%%%
% FOLIENSTIL: Standard
% !!!ÄNDERUNG HIER:!!!
\PraesentationMasterStandard

\PraesentationTitelseite % Fügt die Startseite ein

\begin{frame}
	\frametitle{Hausaufgabe 1}
	\vspace{0.5cm}

	Beim konzeptuellen Entwurf hat man gewisse Freiheitsgrade hinsichtlich der Modellierung der realen Welt.
	Unter anderem hat man folgende Alternativen, die Sie an unserem Universitätsschema beispielhaft illustrieren sollten:

	\begin{itemize}
		\item Man kann ternäre Beziehungen in binäre Beziehungen transformieren.

		      Betrachten Sie dazu die Beziehung \textit{prüfen} und erläutern Sie die Vor- und Nachteile
		      einer solchen Transformation.
		\item Man hat manchmal die Wahl, ein Konzept der realen Welt als Beziehung
		      oder als Entitytyp zu modellieren. Erörtern Sie dies wiederum am Beispiel
		      der Beziehung \textit{prüfen} im Gegensatz zu einem eigenständigen Entitytyp \textit{Prüfungen}.
		\item Ein Konzept der realen Welt kann manchmal als Entitytyp mit zugehörigem
		      Beziehungstyp und manchmal als Attribut dargestellt werden.
		      Ein Beispiel hierfür ist das Attribut Raum des Entitytyps
		      \textit{Professoren} im bekannten Uni Schema. Diskutieren Sie die Alternativen.
	\end{itemize}
\end{frame}

\begin{frame}
	\frametitle{Hausaufgabe 2}
	\vspace{0.5cm}

	Formulieren Sie folgende Anfragen auf dem bekannten Universitätsschema in der relationalen Algebra:
	\begin{enumerate}[a)]
		\item Finden Sie alle \textit{Vorlesungen}, die keine Hörer haben.
		\item Finden Sie alle \textit{Studenten}, die alle \textit{Vorlesungen} hören.
	\end{enumerate}
\end{frame}

\begin{frame}
	\frametitle{Gruppenaufgabe 1}
	\vspace{0.5cm}

	\begin{multicols}{2}
		Gegeben sei die ER-Modellierung von Zugverbindungen in der Abbildung.
		Beachten Sie: verbindet modelliert ein Teilstück einer Verbindung,
		d.h. auf der Strecke München \( \rightarrow \) Hamburg gibt es einen Eintrag für
		die Teilstrecke von München nach Nürnberg,
		einen Eintrag für Nürnberg nach Würzburg,
		einen Eintrag für die Teilstrecke Würzburg nach Göttingen und einen Eintrag von Göttingen nach Hamburg.
		\begin{enumerate}[a)]
			\item<1-> Fügen Sie bei den Beziehungen Funktionalitätsangaben hinzu.
			\item<2-> Übertragen Sie das ER-Modell in ein relationales Schema.
			\item<3-> Verfeinern Sie das relationale Schema soweit möglich durch Eliminierung von Relationen.
		\end{enumerate}

		\vfill\columnbreak
		\begin{center}
			\includegraphics[height=.5\paperheight]{./er_modell1.png}
		\end{center}
	\end{multicols}
\end{frame}


\begin{frame}
	\frametitle{Gruppenaufgabe 2}
	\vspace{0.5cm}

	iBike ist ein neuer Anbieter von free-floating bike sharing in München,
	d.h. die Kunden können die Fahrräder an
	beliebigen Orten im Stadtgebiet ausleihen und auch wieder abstellen.
	Modellieren Sie das iBike-System, das \textit{Kunden}, \textit{Fahrräder} und
	\textit{individuelle Fahrten} (von wann bis wann, von wo nach wo, Preis) verwaltet,
	als ER-Diagramm. Tragen Sie min-max und Funktionalitätsangaben ein.

	Stellen Sie die folgenden Anfragen an das iBike System in relationaler Algebra:
	\begin{enumerate}[a)]
		\item Welche Fahrräder wurden noch nie gefahren?
		\item Was ist der letzte bekannte Standort vom Fahrrad mit der Kennung '123'?
		\item Geben Sie eine Liste der Fahrräder aus, mit denen der Kunde 'Alfons Kemper' schon gefahren ist.
		\item Geben Sie eine Liste der Fahrräder aus, mit denen der Kunde ’Alfons Kemper’ \textbf{noch nicht} gefahren ist.
		\item Wer ist schon mit \textbf{allen} Fahrrädern gefahren?
		\item Am Standort '48.1718164, 11.5510174' wurden sehr viele Fahrräder abgestellt
		      (das ist mitten im Olympiasee). iBike möchte wissen,
		      ob jemand mehr als ein Fahrrad dort abgestellt hat.
		      Dann ist dieser Kunde vermutlich ein Vandale.
	\end{enumerate}
\end{frame}

%%%%%%%%%%%%%%%%%%%%%%%%%%%%%%%%%%%%%%%%%%%%%%%%%%%%%%%%%%%%%%%%%%%%%%%%%%%%%%%%
\end{document} % !!! NICHT ENTFERNEN !!!
%%%%%%%%%%%%%%%%%%%%%%%%%%%%%%%%%%%%%%%%%%%%%%%%%%%%%%%%%%%%%%%%%%%%%%%%%%%%%%%%
