%%%%%%%%%%%%%%%%%%%%%%%%%%%%%%%%%%%%%%%%%%%%%%%%%%%%%%%%%%%%%%%%%%%%%%%%%%%%%%%%
% TUM-Vorlage: Präsentation
%%%%%%%%%%%%%%%%%%%%%%%%%%%%%%%%%%%%%%%%%%%%%%%%%%%%%%%%%%%%%%%%%%%%%%%%%%%%%%%%
%
% Rechteinhaber:
%     Technische Universität München
%     https://www.tum.de
% 
% Gestaltung:
%     ediundsepp Gestaltungsgesellschaft, München
%     http://www.ediundsepp.de
% 
% Technische Umsetzung:
%     eWorks GmbH, Frankfurt am Main
%     http://www.eworks.de
%
%%%%%%%%%%%%%%%%%%%%%%%%%%%%%%%%%%%%%%%%%%%%%%%%%%%%%%%%%%%%%%%%%%%%%%%%%%%%%%%%


%%%%%%%%%%%%%%%%%%%%%%%%%%%%%%%%%%%%%%%%%%%%%%%%%%%%%%%%%%%%%%%%%%%%%%%%%%%%%%%%
% Zur Wahl des Seitenverhältnisses bitte einen der beiden folgenden Befehle
% auskommentieren und den ausführen lassen:
%\documentclass[aspectratio=169]{beamer}
\documentclass[t,aspectratio=169]{beamer}
\usepackage[
	orientation=landscape,
	size=custom,
	width=25.4,
	height=14.2875,
	scale=0.5
]{beamerposter}
\usepackage{verbatim}
\usepackage{booktabs}
\usepackage{verbatim}
\usepackage[cache=false]{minted}
\usemintedstyle{tango}
\definecolor{codeBG}{rgb}{0.97, 0.97, 0.99}
\setminted{linenos=true, bgcolor=codeBG, mathescape=true}

\newcommand{\PraesentationSchriftgroesseSehrGross}{\fontsize{25}{38}}
\newcommand{\PraesentationSchriftgroesseGross}{\fontsize{18}{27}}
\newcommand{\PraesentationSchriftgroesseNormal}{\fontsize{14}{21}}
\newcommand{\PraesentationSchriftgroesseKlein}{\fontsize{11}{17}}
\newcommand{\PraesentationSchriftgroesseDreizeiler}{\fontsize{7}{10}}
\newcommand{\PraesentationSchriftgroesseAufzaehlungszeichen}{\fontsize{10}{8}}

\newcommand{\PraesentationAbstandAbsatz}{18pt}
\newcommand{\PraesentationPositionKorrekturOben}{-1cm}
\newcommand{\PraesentationBeispieleSchriftgroessen}{25 | 18 | 14 | 11}

%% Join Operator declarations
\usepackage{amsmath}
\usepackage{amssymb}
\usepackage{ifsym}

\def\ojoin{\setbox0=\hbox{$\bowtie$}%
    \rule[-.02ex]{.25em}{.4pt}\llap{\rule[\ht0]{.25em}{.4pt}}}

\def\leftouterjoin{\mathbin{\ojoin\mkern-5.8mu\bowtie}}
\def\rightouterjoin{\mathbin{\bowtie\mkern-5.8mu\ojoin}}
\def\fullouterjoin{\mathbin{\ojoin\mkern-5.8mu\bowtie\mkern-5.8mu\ojoin}}
\DeclareMathOperator*{\join}{\bowtie}
\DeclareMathOperator*{\leftsemijoin}{\ltimes}
\DeclareMathOperator*{\rightsemijoin}{\rtimes}
\DeclareMathOperator*{\rightantijoin}{\rhd}
\DeclareMathOperator*{\leftantijoin}{\lhd}

\input{../template/res/Praeambel.tex} % Seitenverhältnis 16:9

%%%%%%%%%%%%%%%%%%%%%%%%%%%%%%%%%%%%%%%%%%%%%%%%%%%%%%%%%%%%%%%%%%%%%%%%%%%%%%%%


%%%%%%%%%%%%%%%%%%%%%%%%%%%%%%%%%%%%%%%%%%%%%%%%%%%%%%%%%%%%%%%%%%%%%%%%%%%%%%%%
%%%%%%%%%%%%%%%%%%%%%%%%%%%%%%%%%%%%%%%%%%%%%%%%%%%%%%%%%%%%%%%%%%%%%%%%%%%%%%%%
% TUM-Vorlage: Personenspezifische Informationen
%%%%%%%%%%%%%%%%%%%%%%%%%%%%%%%%%%%%%%%%%%%%%%%%%%%%%%%%%%%%%%%%%%%%%%%%%%%%%%%%
%
% Rechteinhaber:
%     Technische Universität München
%     https://www.tum.de
% 
% Gestaltung:
%     ediundsepp Gestaltungsgesellschaft, München
%     http://www.ediundsepp.de
% 
% Technische Umsetzung:
%     eWorks GmbH, Frankfurt am Main
%     http://www.eworks.de
%
%%%%%%%%%%%%%%%%%%%%%%%%%%%%%%%%%%%%%%%%%%%%%%%%%%%%%%%%%%%%%%%%%%%%%%%%%%%%%%%%

% Für die Person anpassen:

\newcommand{\PersonTitel}{}
\newcommand{\PersonVorname}{Max}
\newcommand{\PersonNachname}{Frühauf}
\newcommand{\PersonStadt}{@Ort@}
\newcommand{\PersonAdresse}{%
    @Adresse@\\%
    @Plz@~\PersonStadt%
}
\newcommand{\PersonTelefon}{@Telefon@}
\newcommand{\PersonFax}{@Fax@}
\newcommand{\PersonEmail}{max.fruehauf@tum.de}
\newcommand{\PersonWebseite}{@Web@}

\newcommand{\FakultaetAnsprechpartner}{@Ansprechpartner@}
% Fakultät:
\newcommand{\FakultaetName}{Fakultät für Informatik}
\newcommand{\LehrstuhlName}{@Lehrstuhlname@}
% Musterdaten:
\newcommand{\EinstellungBankName}{Bayerische Landesbank}
\newcommand{\EinstellungBankIBAN}{DE10700500000000024866}
\newcommand{\EinstellungBankBIC}{BYLADEMM}
\newcommand{\EinstellungSteuernummer}{143/241/80037}
\newcommand{\EinstellungUmsatzsteuerIdentifikationsnummer}{DE811193231}

\hyphenation{} % eigene Silbentrennung                    % !!! DATEI ANPASSEN !!!
%%%%%%%%%%%%%%%%%%%%%%%%%%%%%%%%%%%%%%%%%%%%%%%%%%%%%%%%%%%%%%%%%%%%%%%%%%%%%%%%


% \renewcommand{\PersonTitel}{Dr. rer. nat.}
\newcommand{\Datum}{\today}

\renewcommand{\PraesentationFusszeileZusatz}{| Tutorium Grundlagen: Datenbanken WS 18/19}

\title{Tutorübung 4}
\author{\PersonVorname{} \PersonNachname}
\institute[]{\UniversitaetName \\ \FakultaetName}
\date[\Datum]{15. Oktober 2018}


%%%%%%%%%%%%%%%%%%%%%%%%%%%%%%%%%%%%%%%%%%%%%%%%%%%%%%%%%%%%%%%%%%%%%%%%%%%%%%%%
\input{../template/res/Anfang.tex} % !!! NICHT ENTFERNEN !!!
%%%%%%%%%%%%%%%%%%%%%%%%%%%%%%%%%%%%%%%%%%%%%%%%%%%%%%%%%%%%%%%%%%%%%%%%%%%%%%%%
\begin{document}
\setlength{\baselineskip}{\PraesentationAbstandAbsatz}
\setlength{\parskip}{\baselineskip}

%%%%%%%%%%%%%%%%%%%%%%%%%%%%%%%%%%%%%%%%%%%%%%%%%%%%%%%%%%%%%%%%%%%%%%%%%%%%%%%%
% FOLIENSTIL: Standard
% !!!ÄNDERUNG HIER:!!!
\PraesentationMasterStandard

\PraesentationTitelseite % Fügt die Startseite ein

\begin{frame}
	\frametitle{Hausaufgabe 1}
	\vspace{0.25cm}

	Formulieren Sie folgende Anfragen auf dem bekannten Universitätsschema in der Relationenalgebra:
	\begin{enumerate}[a)]
		\item Finden Sie die \textit{Assistenten} vom \textit{Professoren}, die den Stundenten Fichte unterrichtet haben -
		      z.B. als potentielle Betreuer seiner Diplomarbeit.
		\item Finden Sie die \textit{Studenten}, die \textit{Vorlesungen} hören (bzw. gehört haben), für die ihnen
		      die direkten Voraussetzungen fehlen.
	\end{enumerate}
\end{frame}

\begin{frame}
	\frametitle{Hausaufgabe 2}
	\vspace{0.25cm}

	Formulieren Sie die folgenden Anfragen auf dem bekannten Universitätsschema im
	relationalen \textbf{Tupelkalkül} und im relationalen \textbf{Domänenkalkül}:
	\begin{enumerate}[a)]
		\item Geben Sie alle \textit{Vorlesungen} an, die der \textit{Student} Xenokrates gehört hat.
		\item Geben Sie die Titel der direkten Voraussetzungen für die \textit{Vorlesung} Wissenschaftstheorie an.
		\item Geben Sie Paare von \textit{Studenten}(-Namen) an, die sich aus der \textit{Vorlesung} Grundzüge kennen.
	\end{enumerate}
\end{frame}

\begin{frame}
	\frametitle{Hausaufgabe 3}
	\vspace{0.25cm}

	Gegeben sei die folgende Relation \textbf{Zehnkampf} mit Athletennamen und den von ihnen erreichten Punkten im Zehnkampf:

	\begin{enumerate}[a)]
		\item Ermitteln Sie die Goldmedaillengewinner in relationaler Algebra. \\
		      (Eine Goldmedaille bekommen alle, für die gilt: es gibt niemand besseren (also mit mehr Punkten).)
		\item Ermitteln Sie die Silbermedaillengewinner im Tupelkalkül. \\
		      (Eine Silbermedaille bekommen alle, für die gilt: es gibt genau eine/n bessere/n.)
	\end{enumerate}
	\begin{table}[]
		\begin{tabular}{c|c}
			Name         & Punkte       \\ \hline
			Eaton        & 8869         \\
			Suarez       & 8523         \\
			Behrenbruch  & 8126         \\
			Hardee       & 8671         \\
			\( \hdots \) & \( \hdots \)
		\end{tabular}
	\end{table}
\end{frame}

\begin{frame}
	\frametitle{Hausaufgabe 4}
	\vspace{0.25cm}

	Gegeben seien die beiden Relationen \( R: {[a_1, \hdots, a_n]} \) und \( S: {[b_1, \hdots, b_m]} \). \\
	Geben Sie die folgenden Ausdrücke im Tupel- und Domänenkalkül an:
	\begin{enumerate}[a)]
		\item \( Q_1 := R \join_{a_1 = b_1} S \)
		\item \( Q_1 := R \leftouterjoin_{a_1 = b_1} S \)
		\item \( Q_1 := R \rightsemijoin_{a_1 = b_1} S \)
		\item \( Q_1 := R \leftantijoin_{a_1 = b_1} S \)
	\end{enumerate}
\end{frame}

\begin{frame}[fragile]
	\frametitle{Quiz 1}
	\vspace{0.25cm}

	\begin{multicols}{2}
		Erweitern Sie die Uni-Datenbank um die Relation:

		\[ \text{StudienPlan}: \{ [\text{Semester}, \text{VorlesungsNr}] \} \]

		\begin{enumerate}[a)]
			\item<1-> Bestimmen Sie in relationaler Algebra die Studenten, die alle für ihr Semester
				vorgesehenen Vorlesungen hören.
			\item<3-> Bestimmen Sie in SQL die Studenten,
				die nur Vorlesungen ihres Semesters hören (nicht notwendigerweise alle).
		\end{enumerate}
		\vfill\columnbreak
		\pause

		Lösung in Operatorbaumdarstellung:
		\begin{center}
			\includegraphics[height=.3\paperheight]{./quiz1.png}
		\end{center}
	\end{multicols}

\end{frame}

\begin{frame}[fragile]
	\frametitle{Quiz 1}
	\vspace{0.25cm}

	\begin{multicols}{2}
		Lösung:
		\begin{minted}{sql}
SELECT *
FROM studenten s
WHERE NOT EXISTS (
	SELECT *
	FROM hoeren h, studienplan sp
	WHERE s.matrnr = h.matrnr
	  AND h.vorlnr = sp.vorlnr
	  AND s.semester <> sp.semester
) AND EXISTS (
	SELECT *
	FROM hoeren h
	WHERE h.marnr = s.matrnr
);
		\end{minted}
		\vfill\columnbreak
		\pause
		
		Alternativlösung mit 'all':
		\begin{minted}{sql}
SELECT *
FROM studenten s
WHERE s.semester = all(
	SELECT sp.semester
	FROM hoeren h, studienplan sp
	WHERE s.matrnr = h.matrnr 
	  AND h.vorlnr = sp.vorlnr
) AND EXISTS (
	SELECT *
	FROM hoeren h
	WHERE h.matrnr = s.matrnr
);
		\end{minted}
	\end{multicols}

\end{frame}
%%%%%%%%%%%%%%%%%%%%%%%%%%%%%%%%%%%%%%%%%%%%%%%%%%%%%%%%%%%%%%%%%%%%%%%%%%%%%%%%
\end{document} % !!! NICHT ENTFERNEN !!!
%%%%%%%%%%%%%%%%%%%%%%%%%%%%%%%%%%%%%%%%%%%%%%%%%%%%%%%%%%%%%%%%%%%%%%%%%%%%%%%%
