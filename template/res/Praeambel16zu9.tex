%\documentclass[aspectratio=169]{beamer}
\documentclass[t,aspectratio=169]{beamer}
\usepackage[
	orientation=landscape,
	size=custom,
	width=25.4,
	height=14.2875,
	scale=0.5
]{beamerposter}
\usepackage{verbatim}
\usepackage{booktabs}
\usepackage{verbatim}
\usepackage{color}
\usepackage[cache=false]{minted}
\usemintedstyle{tango}
\definecolor{codeBG}{rgb}{0.97, 0.97, 0.99}
\setminted{linenos=true, bgcolor=codeBG, mathescape=true}

\newcommand{\PraesentationSchriftgroesseSehrGross}{\fontsize{25}{38}}
\newcommand{\PraesentationSchriftgroesseGross}{\fontsize{18}{27}}
\newcommand{\PraesentationSchriftgroesseNormal}{\fontsize{14}{21}}
\newcommand{\PraesentationSchriftgroesseKlein}{\fontsize{11}{17}}
\newcommand{\PraesentationSchriftgroesseDreizeiler}{\fontsize{7}{10}}
\newcommand{\PraesentationSchriftgroesseAufzaehlungszeichen}{\fontsize{10}{8}}

\newcommand{\PraesentationAbstandAbsatz}{18pt}
\newcommand{\PraesentationPositionKorrekturOben}{-1cm}
\newcommand{\PraesentationBeispieleSchriftgroessen}{25 | 18 | 14 | 11}

%% Join Operator declarations
\usepackage{amsmath}
\usepackage{amssymb}
\usepackage{ifsym}

\def\ojoin{\setbox0=\hbox{$\bowtie$}%
    \rule[-.02ex]{.25em}{.4pt}\llap{\rule[\ht0]{.25em}{.4pt}}}

\def\leftouterjoin{\mathbin{\ojoin\mkern-5.8mu\bowtie}}
\def\rightouterjoin{\mathbin{\bowtie\mkern-5.8mu\ojoin}}
\def\fullouterjoin{\mathbin{\ojoin\mkern-5.8mu\bowtie\mkern-5.8mu\ojoin}}
\DeclareMathOperator*{\join}{\bowtie}
\DeclareMathOperator*{\leftsemijoin}{\ltimes}
\DeclareMathOperator*{\rightsemijoin}{\rtimes}
\DeclareMathOperator*{\rightantijoin}{\rhd}
\DeclareMathOperator*{\leftantijoin}{\lhd}

\usepackage[utf8]{inputenc}
\usepackage[T1]{fontenc} % Zeichensatzkodierung

\usepackage{calc} % Berechnungen

\usepackage[ngerman]{babel} % Deutsche Lokalisierung
\usepackage{graphicx} % Grafiken
\usepackage[export]{adjustbox}
\usepackage[absolute, overlay]{textpos} % Positionierung

% Silbentrennung:
\usepackage{hyphenat}
%\tolerance 2414
%\hbadness 2414
%\emergencystretch 1.5em
%\hfuzz 0.3pt
%\widowpenalty=10000     % Hurenkinder
%\clubpenalty=10000      % Schusterjungen
%\vfuzz \hfuzz

% Euro-Symbol:
\usepackage[gen]{eurosym}
\DeclareUnicodeCharacter{20AC}{\euro{}}

% Schriftart Helvetica:
\usepackage[scaled]{helvet}
\renewcommand{\familydefault}{\sfdefault}

\usepackage{mathptmx} % skalierbare Formelschriften

\usepackage{tabularx}

\usepackage{multicol} % mehrspaltiger Text

\usepackage{tikz}
\usetikzlibrary{arrows, shapes, shapes.multipart, trees, positioning,
    backgrounds, fit, matrix}

% Diagramme:
\usepackage{pgfplots}
\pgfplotsset{compat=default}

% Erweiterbare Fusszeile:
\newcommand{\PraesentationFusszeileZusatz}{}
