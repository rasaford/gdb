%%%%%%%%%%%%%%%%%%%%%%%%%%%%%%%%%%%%%%%%%%%%%%%%%%%%%%%%%%%%%%%%%%%%%%%%%%%%%%%%
% TUM-Vorlage: Präsentation
%%%%%%%%%%%%%%%%%%%%%%%%%%%%%%%%%%%%%%%%%%%%%%%%%%%%%%%%%%%%%%%%%%%%%%%%%%%%%%%%
%
% Rechteinhaber:
%     Technische Universität München
%     https://www.tum.de
% 
% Gestaltung:
%     ediundsepp Gestaltungsgesellschaft, München
%     http://www.ediundsepp.de
% 
% Technische Umsetzung:
%     eWorks GmbH, Frankfurt am Main
%     http://www.eworks.de
%
%%%%%%%%%%%%%%%%%%%%%%%%%%%%%%%%%%%%%%%%%%%%%%%%%%%%%%%%%%%%%%%%%%%%%%%%%%%%%%%%


%%%%%%%%%%%%%%%%%%%%%%%%%%%%%%%%%%%%%%%%%%%%%%%%%%%%%%%%%%%%%%%%%%%%%%%%%%%%%%%%
% Zur Wahl des Seitenverhältnisses bitte einen der beiden folgenden Befehle
% auskommentieren und den ausführen lassen:
%\documentclass[aspectratio=169]{beamer}
\documentclass[t,aspectratio=169]{beamer}
\usepackage[
	orientation=landscape,
	size=custom,
	width=25.4,
	height=14.2875,
	scale=0.5
]{beamerposter}
\usepackage{verbatim}
\usepackage{booktabs}
\usepackage{verbatim}
\usepackage[cache=false]{minted}
\usemintedstyle{tango}
\definecolor{codeBG}{rgb}{0.97, 0.97, 0.99}
\setminted{linenos=true, bgcolor=codeBG, mathescape=true}

\newcommand{\PraesentationSchriftgroesseSehrGross}{\fontsize{25}{38}}
\newcommand{\PraesentationSchriftgroesseGross}{\fontsize{18}{27}}
\newcommand{\PraesentationSchriftgroesseNormal}{\fontsize{14}{21}}
\newcommand{\PraesentationSchriftgroesseKlein}{\fontsize{11}{17}}
\newcommand{\PraesentationSchriftgroesseDreizeiler}{\fontsize{7}{10}}
\newcommand{\PraesentationSchriftgroesseAufzaehlungszeichen}{\fontsize{10}{8}}

\newcommand{\PraesentationAbstandAbsatz}{18pt}
\newcommand{\PraesentationPositionKorrekturOben}{-1cm}
\newcommand{\PraesentationBeispieleSchriftgroessen}{25 | 18 | 14 | 11}

%% Join Operator declarations
\usepackage{amsmath}
\usepackage{amssymb}
\usepackage{ifsym}

\def\ojoin{\setbox0=\hbox{$\bowtie$}%
    \rule[-.02ex]{.25em}{.4pt}\llap{\rule[\ht0]{.25em}{.4pt}}}

\def\leftouterjoin{\mathbin{\ojoin\mkern-5.8mu\bowtie}}
\def\rightouterjoin{\mathbin{\bowtie\mkern-5.8mu\ojoin}}
\def\fullouterjoin{\mathbin{\ojoin\mkern-5.8mu\bowtie\mkern-5.8mu\ojoin}}
\DeclareMathOperator*{\join}{\bowtie}
\DeclareMathOperator*{\leftsemijoin}{\ltimes}
\DeclareMathOperator*{\rightsemijoin}{\rtimes}
\DeclareMathOperator*{\rightantijoin}{\rhd}
\DeclareMathOperator*{\leftantijoin}{\lhd}

\input{../template/res/Praeambel.tex} % Seitenverhältnis 16:9

%%%%%%%%%%%%%%%%%%%%%%%%%%%%%%%%%%%%%%%%%%%%%%%%%%%%%%%%%%%%%%%%%%%%%%%%%%%%%%%%


%%%%%%%%%%%%%%%%%%%%%%%%%%%%%%%%%%%%%%%%%%%%%%%%%%%%%%%%%%%%%%%%%%%%%%%%%%%%%%%%
%%%%%%%%%%%%%%%%%%%%%%%%%%%%%%%%%%%%%%%%%%%%%%%%%%%%%%%%%%%%%%%%%%%%%%%%%%%%%%%%
% TUM-Vorlage: Personenspezifische Informationen
%%%%%%%%%%%%%%%%%%%%%%%%%%%%%%%%%%%%%%%%%%%%%%%%%%%%%%%%%%%%%%%%%%%%%%%%%%%%%%%%
%
% Rechteinhaber:
%     Technische Universität München
%     https://www.tum.de
% 
% Gestaltung:
%     ediundsepp Gestaltungsgesellschaft, München
%     http://www.ediundsepp.de
% 
% Technische Umsetzung:
%     eWorks GmbH, Frankfurt am Main
%     http://www.eworks.de
%
%%%%%%%%%%%%%%%%%%%%%%%%%%%%%%%%%%%%%%%%%%%%%%%%%%%%%%%%%%%%%%%%%%%%%%%%%%%%%%%%

% Für die Person anpassen:

\newcommand{\PersonTitel}{}
\newcommand{\PersonVorname}{Max}
\newcommand{\PersonNachname}{Frühauf}
\newcommand{\PersonStadt}{@Ort@}
\newcommand{\PersonAdresse}{%
    @Adresse@\\%
    @Plz@~\PersonStadt%
}
\newcommand{\PersonTelefon}{@Telefon@}
\newcommand{\PersonFax}{@Fax@}
\newcommand{\PersonEmail}{max.fruehauf@tum.de}
\newcommand{\PersonWebseite}{@Web@}

\newcommand{\FakultaetAnsprechpartner}{@Ansprechpartner@}
% Fakultät:
\newcommand{\FakultaetName}{Fakultät für Informatik}
\newcommand{\LehrstuhlName}{@Lehrstuhlname@}
% Musterdaten:
\newcommand{\EinstellungBankName}{Bayerische Landesbank}
\newcommand{\EinstellungBankIBAN}{DE10700500000000024866}
\newcommand{\EinstellungBankBIC}{BYLADEMM}
\newcommand{\EinstellungSteuernummer}{143/241/80037}
\newcommand{\EinstellungUmsatzsteuerIdentifikationsnummer}{DE811193231}

\hyphenation{} % eigene Silbentrennung                    % !!! DATEI ANPASSEN !!!
%%%%%%%%%%%%%%%%%%%%%%%%%%%%%%%%%%%%%%%%%%%%%%%%%%%%%%%%%%%%%%%%%%%%%%%%%%%%%%%%


% \renewcommand{\PersonTitel}{Dr. rer. nat.}
\newcommand{\Datum}{\today}

\renewcommand{\PraesentationFusszeileZusatz}{| Tutorium Grundlagen: Datenbanken WS 18/19}

\title{Tutorübung 6}
\author{\PersonVorname{} \PersonNachname}
\institute[]{\UniversitaetName \\ \FakultaetName}
\date[\Datum]{15. Oktober 2018}


%%%%%%%%%%%%%%%%%%%%%%%%%%%%%%%%%%%%%%%%%%%%%%%%%%%%%%%%%%%%%%%%%%%%%%%%%%%%%%%%
\input{../template/res/Anfang.tex} % !!! NICHT ENTFERNEN !!!
%%%%%%%%%%%%%%%%%%%%%%%%%%%%%%%%%%%%%%%%%%%%%%%%%%%%%%%%%%%%%%%%%%%%%%%%%%%%%%%%
\begin{document}
\setlength{\baselineskip}{\PraesentationAbstandAbsatz}
\setlength{\parskip}{\baselineskip}

%%%%%%%%%%%%%%%%%%%%%%%%%%%%%%%%%%%%%%%%%%%%%%%%%%%%%%%%%%%%%%%%%%%%%%%%%%%%%%%%
% FOLIENSTIL: Standard
% !!!ÄNDERUNG HIER:!!!
\PraesentationMasterStandard

\PraesentationTitelseite % Fügt die Startseite ein

\begin{frame}
	\frametitle{Hausaufgabe 1}
	\vspace{0.25cm}

	\begin{multicols}{2}
		"Fleißige Studenten": Formulieren Sie eine SQL-Anfrage, 
		um die Studenten zu ermitteln, die mehr SWS belegt haben als der Durchschnitt. 
		Berücksichtigen Sie dabei auch Totalverweigerer, die gar keine Vorlesungen hören.
		\vfill\columnbreak

		\begin{center}
			\includegraphics[height=.6\paperheight]{../img/uni.pdf}
		\end{center}
	\end{multicols}
\end{frame}

\begin{frame}[fragile]
	\frametitle{Hausaufgabe 1}
	\vspace{0.25cm}

	\begin{multicols}{2}
		"Fleißige Studenten": Formulieren Sie eine SQL-Anfrage, 
		um die Studenten zu ermitteln, die mehr SWS belegt haben als der Durchschnitt. 
		Berücksichtigen Sie dabei auch Totalverweigerer, die gar keine Vorlesungen hören.
		\begin{minted}{sql}
WITH GesamtSWS AS (
	SELECT sum(SWS * 1.00) AS AnzSWS
	FROM hoeren h, Vorlesungen v
	WHERE v.VorlNr = h.VorlNr
),
GesamtStudenten AS (
	SELECT count(MatrNr) AS AnzStudenten
	FROM Studenten
)
		\end{minted}
		\vfill\columnbreak

		\begin{minted}{sql}
SELECT *
FROM Studenten s
WHERE s.MatrNr in (
	SELECT h.MatrNr
	FROM hoeren h, Vorlesungen v
	WHERE h.VorlNr = v.VorlNr
	GROUP BY h.MatrNr
	HAVING sum(SWS) > (
		SELECT AnzSWS / AnzStudenten
		FROM GesamtSWS, GesamtStudenten
	)
);
		\end{minted}

	\end{multicols}
\end{frame}

\begin{frame}[fragile]
	\frametitle{Hausaufgabe 2}
	\vspace{0.25cm}

	\begin{multicols}{2}
		Was bringt der Vorlesungsbesuch? Finden Sie heraus, ob es für Prüfungen von Vorteil ist, 
		die jeweiligen Vorlesungen auch gehört zu haben. 
		
		Ermitteln Sie dazu die Durchschnittsnote der Prüfungen, zu denen die Studenten 
		die Vorlesungen nicht gehört haben und die Durchschnittsnote der 
		Prüfungen, zu denen sie die Vorlesungen gehört haben. \\
		- Formulieren Sie Ihre Antwort in SQL
		\vfill\columnbreak

		\begin{center}
			\includegraphics[height=.6\paperheight]{../img/uni.pdf}
		\end{center}
	\end{multicols}
\end{frame}

\begin{frame}[fragile]
	\frametitle{Hausaufgabe 2}
	\vspace{0.25cm}

	\begin{multicols}{2}
		\begin{minted}{sql}
WITH nichtGehoert AS (
	SELECT avg(Note) as DnoteVLNichtGehoert
	FROM pruefen p
	WHERE NOT EXISTS (
		SELECT *
		FROM hoeren h
		WHERE h.VorlNr = p.VorlNr
		  AND h.MatrNr = p.MatrNr
	)
),
		\end{minted}
		\vfill\columnbreak

		\begin{minted}{sql}
gehoert AS (
	SELECT avg(Note) AS DnoteVLGehoert
	FROM pruefen p
	WHERE EXISTS (
		SELECT *
		FROM hoeren h
		WHERE h.VorlNr = p.VorlNr
		  AND h.MatrNr = p.MatrNr
	)
)
SELECT *
FROM nichtGehoert, gehoert;
		\end{minted}
	\end{multicols}
\end{frame}


\begin{frame}[fragile]
\frametitle{Hausaufgabe 3}
\vspace{0.25cm}

\begin{multicols}{2}
	Welche Studenten haben alle Vorlesungen, 
	die sie haben prüfen lassen, auch tatsächlich vorher gehört?

	Tipp: \textit{'Es darf keine Vorlesung geben, die geprüft wurde, 
	zu der es aber keinen Eintrag in hören gibt.'}
	\vfill\columnbreak

	\begin{center}
		\includegraphics[height=.6\paperheight]{../img/uni.pdf}
	\end{center}
\end{multicols}
\end{frame}

\begin{frame}[fragile]
\frametitle{Hausaufgabe 3}
\vspace{0.25cm}

\begin{multicols}{2}
	Welche Studenten haben alle Vorlesungen, 
	die sie haben prüfen lassen, auch tatsächlich vorher gehört?

	Tipp: \textit{'Es darf keine Vorlesung geben, die geprüft wurde, 
	zu der es aber keinen Eintrag in hören gibt.'}
	\vfill\columnbreak

	\begin{minted}{sql}
SELECT *
FROM Studenten s
WHERE NOT EXISTS (
	SELECT *
	FROM pruefen p
	WHERE s.MatrNr = p.MatrNr
	  AND NOT EXISTS (
		  SELECT *
		  FROM hoeren h
		  WHERE h.MatrNr = s.MatrNr
		    AND h.VorlNr = p.VorlNr
	  )
);
	\end{minted}
\end{multicols}
\end{frame}

\begin{frame}[fragile]
\frametitle{Hausaufgabe 4}
\vspace{0.25cm}

\begin{multicols}{2}
	"Frühestes Semester": Formulieren Sie eine SQL-Anfrage, 
	um das Semester zu ermitteln in dem die Vorlesung "Der Wiener Kreis" frühestens gehört 
	werden kann. Testen Sie die Anfrage auch mit anderen Vorlesungen, 
	insbesondere mit "Logik".
	\vfill\columnbreak
	
	\begin{center}
		\includegraphics[height=.6\paperheight]{../img/uni.pdf}
	\end{center}
\end{multicols}
\end{frame}

\begin{frame}[fragile]
\frametitle{Hausaufgabe 4}
\vspace{0.25cm}

\begin{multicols}{2}
	"Frühestes Semester": Formulieren Sie eine SQL-Anfrage, 
	um das Semester zu ermitteln in dem die Vorlesung "Der Wiener Kreis" frühestens gehört 
	werden kann. Testen Sie die Anfrage auch mit anderen Vorlesungen, 
	insbesondere mit "Logik".
	\vfill\columnbreak

	\begin{minted}{sql}
WITH RECURSIVE voraussetzenRek(vorlNr, counter) AS (
	SELECT v.VorlNr, 1 as counter
	FROM Vorlesungen v
	WHERE v.Titel = 'Der Wiener Kreis'
) UNION (
	SELECT vs.vorgaenger, vsr.counter + 1 
	                               AS counter
	FROM voraussetzenRek vsr, voraussetzen vs
	WHERE vsr.VorlNr = vs.Nachfolger
)
SELECT max(counter) as fruehestesSemester
FROM voraussetzenRek
	\end{minted}
\end{multicols}
\end{frame}
%%%%%%%%%%%%%%%%%%%%%%%%%%%%%%%%%%%%%%%%%%%%%%%%%%%%%%%%%%%%%%%%%%%%%%%%%%%%%%%%
\end{document} % !!! NICHT ENTFERNEN !!!
%%%%%%%%%%%%%%%%%%%%%%%%%%%%%%%%%%%%%%%%%%%%%%%%%%%%%%%%%%%%%%%%%%%%%%%%%%%%%%%%
