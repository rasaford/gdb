%%%%%%%%%%%%%%%%%%%%%%%%%%%%%%%%%%%%%%%%%%%%%%%%%%%%%%%%%%%%%%%%%%%%%%%%%%%%%%%%
% TUM-Vorlage: Präsentation
%%%%%%%%%%%%%%%%%%%%%%%%%%%%%%%%%%%%%%%%%%%%%%%%%%%%%%%%%%%%%%%%%%%%%%%%%%%%%%%%
%
% Rechteinhaber:
%     Technische Universität München
%     https://www.tum.de
% 
% Gestaltung:
%     ediundsepp Gestaltungsgesellschaft, München
%     http://www.ediundsepp.de
% 
% Technische Umsetzung:
%     eWorks GmbH, Frankfurt am Main
%     http://www.eworks.de
%
%%%%%%%%%%%%%%%%%%%%%%%%%%%%%%%%%%%%%%%%%%%%%%%%%%%%%%%%%%%%%%%%%%%%%%%%%%%%%%%%


%%%%%%%%%%%%%%%%%%%%%%%%%%%%%%%%%%%%%%%%%%%%%%%%%%%%%%%%%%%%%%%%%%%%%%%%%%%%%%%%
% Zur Wahl des Seitenverhältnisses bitte einen der beiden folgenden Befehle
% auskommentieren und den ausführen lassen:
%\documentclass[aspectratio=169]{beamer}
\documentclass[t,aspectratio=169]{beamer}
\usepackage[
	orientation=landscape,
	size=custom,
	width=25.4,
	height=14.2875,
	scale=0.5
]{beamerposter}
\usepackage{verbatim}
\usepackage{booktabs}
\usepackage{verbatim}
\usepackage[cache=false]{minted}
\usemintedstyle{tango}
\definecolor{codeBG}{rgb}{0.97, 0.97, 0.99}
\setminted{linenos=true, bgcolor=codeBG, mathescape=true}

\newcommand{\PraesentationSchriftgroesseSehrGross}{\fontsize{25}{38}}
\newcommand{\PraesentationSchriftgroesseGross}{\fontsize{18}{27}}
\newcommand{\PraesentationSchriftgroesseNormal}{\fontsize{14}{21}}
\newcommand{\PraesentationSchriftgroesseKlein}{\fontsize{11}{17}}
\newcommand{\PraesentationSchriftgroesseDreizeiler}{\fontsize{7}{10}}
\newcommand{\PraesentationSchriftgroesseAufzaehlungszeichen}{\fontsize{10}{8}}

\newcommand{\PraesentationAbstandAbsatz}{18pt}
\newcommand{\PraesentationPositionKorrekturOben}{-1cm}
\newcommand{\PraesentationBeispieleSchriftgroessen}{25 | 18 | 14 | 11}

%% Join Operator declarations
\usepackage{amsmath}
\usepackage{amssymb}
\usepackage{ifsym}

\def\ojoin{\setbox0=\hbox{$\bowtie$}%
    \rule[-.02ex]{.25em}{.4pt}\llap{\rule[\ht0]{.25em}{.4pt}}}

\def\leftouterjoin{\mathbin{\ojoin\mkern-5.8mu\bowtie}}
\def\rightouterjoin{\mathbin{\bowtie\mkern-5.8mu\ojoin}}
\def\fullouterjoin{\mathbin{\ojoin\mkern-5.8mu\bowtie\mkern-5.8mu\ojoin}}
\DeclareMathOperator*{\join}{\bowtie}
\DeclareMathOperator*{\leftsemijoin}{\ltimes}
\DeclareMathOperator*{\rightsemijoin}{\rtimes}
\DeclareMathOperator*{\rightantijoin}{\rhd}
\DeclareMathOperator*{\leftantijoin}{\lhd}

\input{../template/res/Praeambel.tex} % Seitenverhältnis 16:9

%%%%%%%%%%%%%%%%%%%%%%%%%%%%%%%%%%%%%%%%%%%%%%%%%%%%%%%%%%%%%%%%%%%%%%%%%%%%%%%%


%%%%%%%%%%%%%%%%%%%%%%%%%%%%%%%%%%%%%%%%%%%%%%%%%%%%%%%%%%%%%%%%%%%%%%%%%%%%%%%%
%%%%%%%%%%%%%%%%%%%%%%%%%%%%%%%%%%%%%%%%%%%%%%%%%%%%%%%%%%%%%%%%%%%%%%%%%%%%%%%%
% TUM-Vorlage: Personenspezifische Informationen
%%%%%%%%%%%%%%%%%%%%%%%%%%%%%%%%%%%%%%%%%%%%%%%%%%%%%%%%%%%%%%%%%%%%%%%%%%%%%%%%
%
% Rechteinhaber:
%     Technische Universität München
%     https://www.tum.de
% 
% Gestaltung:
%     ediundsepp Gestaltungsgesellschaft, München
%     http://www.ediundsepp.de
% 
% Technische Umsetzung:
%     eWorks GmbH, Frankfurt am Main
%     http://www.eworks.de
%
%%%%%%%%%%%%%%%%%%%%%%%%%%%%%%%%%%%%%%%%%%%%%%%%%%%%%%%%%%%%%%%%%%%%%%%%%%%%%%%%

% Für die Person anpassen:

\newcommand{\PersonTitel}{}
\newcommand{\PersonVorname}{Max}
\newcommand{\PersonNachname}{Frühauf}
\newcommand{\PersonStadt}{@Ort@}
\newcommand{\PersonAdresse}{%
    @Adresse@\\%
    @Plz@~\PersonStadt%
}
\newcommand{\PersonTelefon}{@Telefon@}
\newcommand{\PersonFax}{@Fax@}
\newcommand{\PersonEmail}{max.fruehauf@tum.de}
\newcommand{\PersonWebseite}{@Web@}

\newcommand{\FakultaetAnsprechpartner}{@Ansprechpartner@}
% Fakultät:
\newcommand{\FakultaetName}{Fakultät für Informatik}
\newcommand{\LehrstuhlName}{@Lehrstuhlname@}
% Musterdaten:
\newcommand{\EinstellungBankName}{Bayerische Landesbank}
\newcommand{\EinstellungBankIBAN}{DE10700500000000024866}
\newcommand{\EinstellungBankBIC}{BYLADEMM}
\newcommand{\EinstellungSteuernummer}{143/241/80037}
\newcommand{\EinstellungUmsatzsteuerIdentifikationsnummer}{DE811193231}

\hyphenation{} % eigene Silbentrennung                    % !!! DATEI ANPASSEN !!!
%%%%%%%%%%%%%%%%%%%%%%%%%%%%%%%%%%%%%%%%%%%%%%%%%%%%%%%%%%%%%%%%%%%%%%%%%%%%%%%%


% \renewcommand{\PersonTitel}{Dr. rer. nat.}
\newcommand{\Datum}{\today}

\renewcommand{\PraesentationFusszeileZusatz}{| Tutorium Einführung in die Rechnerarchitektur WS 2018/2019}

\title{Tutorübung 2}
\author{\PersonVorname{} \PersonNachname}
\institute[]{\UniversitaetName \\ \FakultaetName}
\date[\Datum]{15. Oktober 2018}


%%%%%%%%%%%%%%%%%%%%%%%%%%%%%%%%%%%%%%%%%%%%%%%%%%%%%%%%%%%%%%%%%%%%%%%%%%%%%%%%
\input{../template/res/Anfang.tex} % !!! NICHT ENTFERNEN !!!
%%%%%%%%%%%%%%%%%%%%%%%%%%%%%%%%%%%%%%%%%%%%%%%%%%%%%%%%%%%%%%%%%%%%%%%%%%%%%%%%
\begin{document}
\setlength{\baselineskip}{\PraesentationAbstandAbsatz}
\setlength{\parskip}{\baselineskip}

%%%%%%%%%%%%%%%%%%%%%%%%%%%%%%%%%%%%%%%%%%%%%%%%%%%%%%%%%%%%%%%%%%%%%%%%%%%%%%%%
% FOLIENSTIL: Standard
% !!!ÄNDERUNG HIER:!!!
\PraesentationMasterStandard

\PraesentationTitelseite % Fügt die Startseite ein

\begin{frame}
	\frametitle{Hausaufgabe 1}
	\vspace{0.5cm}

	\begin{multicols}{2}
		Angenommen,das hier modellierte Übungssystem entspricht dem Übungssystem in Grundlagen:
		Datenbanken. Bestimmen Sie die MinMax Angaben so, dass folgende Einschränkungen
		modelliert werden:
		\begin{itemize}
			\item Ein Tutor hält mindestens eine Übung.
			\item Eine Übung wird von mindestens einem Studenten besucht.
			\item Ein Student kann höchstens eine Übung besuchen.
		\end{itemize}
		Betrachten Sie nun die folgende Ausprägung, die die Beziehung modellieren soll.

		Welche Beziehung besteht zwischen der MinMax Notation und einer solchen
		Ausprägung?

		\vfill\columnbreak

		\begin{table}[]
			\begin{tabular}{c|c|c}
				Name         & GruppenNr    & MatrNr       \\ \hline
				\( \vdots \) & \( \vdots \) & \( \vdots \) \\
				Lang         & G12          & 23           \\
				Passing      & G27          & 42           \\
				Passing      & G27          & 43           \\
				\( \vdots \) & \( \vdots \) & \( \vdots \) \\
				Passing      & G28          & 97           \\
				Passing      & G28          & 98           \\
				Passing      & G28          & 99           \\
				\( \vdots \) & \( \vdots \) & \( \vdots \) \\
			\end{tabular}
		\end{table}

	\end{multicols}
\end{frame}

\begin{frame}
	\frametitle{Hausaufgabe 2}
	\vspace{0.5cm}

	\begin{enumerate}[a)]
		\item Erstellen Sie ein ER-Modell womit sich kausale Zusammenhänge
		      darstellen lassen (Prinzip von Ursache und Wirkung).
		      Nehmen Sie an, dass eine Ursache mehrere Wirkungen haben kann,
		      und dass eine Wirkung auf maximal eine Ursache zurückzuführen ist.
		      Geben Sie die Funktionalitäten an. Verwenden Sie die (min,max)-Notation.
		\item Übertragen Sie das ER-Modell in ein relationales Schema.
		\item Verfeinern Sie das relationale Schema durch Elimination von Relationen.
		\item Formulieren Sie folgende Anfrage in relationaler Algebra jeweils für die Schemas
		      aus den Teilaufgaben b) und c): Finden Sie alle Auswirkungen des Ereignisses mit ID=10.
	\end{enumerate}
\end{frame}

\begin{frame}
	\frametitle{Hausaufgabe 3}
	\vspace{0.5cm}

	Ignorieren Sie die Funktionalitätsangaben und beantworten Sie:
	\begin{itemize}
		\item Wie viele partielle Funktionen der Form \( A \times B \rightarrow C \)
		      können in einer ternären Beziehung auftreten
		      (Ignorieren Sie beim Zählen die Reihenfolge auf der linken Seite der Beziehung).
		\item Nennen Sie alle möglichen partiellen Beziehungen in der hier
		      gezeigten Beziehung "nimmt teil".
		\item Nennen Sie für jede Funktion in Prosa,
		      welche Einschränkung diese darstellt, falls sie gilt.
	\end{itemize}
	Unter Berücksichtigung der Funktionalitätsangaben:
	\begin{itemize}
		\item Welche partiellen Funktionen gelten hier?
	\end{itemize}
\end{frame}

\begin{frame}[fragile]
	\frametitle{Hausaufgabe 4}
	\vspace{0.5cm}

	Formulieren Sie die folgenden Anfragen auf dem bekannten Universitätsschema
	in Relationenalgebra.
	Geben Sie die Lösungen in der Operatorbaum-Darstellung an.
	\begin{enumerate}[a)]
		\item Geben Sie alle \( Vorlesungen \) an, die der Student \( Xenokrates \) gehört hat.
		\item Geben Sie die Titel der direkten Voraussetzungen für die \( Vorlesung \) Wissenschafts-
		      theorie an.
		\item Geben Sie Paare von \( Studenten \)(-Namen) an, die sich aus der \( Vorlesung \) Grundzüge kennen.
	\end{enumerate}

% 	\begin{minted}{sql}
% SELECT *
% FROM Studenten
% WHERE name = 'test'
% 	\end{minted}

\end{frame}
%%%%%%%%%%%%%%%%%%%%%%%%%%%%%%%%%%%%%%%%%%%%%%%%%%%%%%%%%%%%%%%%%%%%%%%%%%%%%%%%
\end{document} % !!! NICHT ENTFERNEN !!!
%%%%%%%%%%%%%%%%%%%%%%%%%%%%%%%%%%%%%%%%%%%%%%%%%%%%%%%%%%%%%%%%%%%%%%%%%%%%%%%%
