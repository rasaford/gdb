%%%%%%%%%%%%%%%%%%%%%%%%%%%%%%%%%%%%%%%%%%%%%%%%%%%%%%%%%%%%%%%%%%%%%%%%%%%%%%%%
% TUM-Vorlage: Präsentation
%%%%%%%%%%%%%%%%%%%%%%%%%%%%%%%%%%%%%%%%%%%%%%%%%%%%%%%%%%%%%%%%%%%%%%%%%%%%%%%%
%
% Rechteinhaber:
%     Technische Universität München
%     https://www.tum.de
% 
% Gestaltung:
%     ediundsepp Gestaltungsgesellschaft, München
%     http://www.ediundsepp.de
% 
% Technische Umsetzung:
%     eWorks GmbH, Frankfurt am Main
%     http://www.eworks.de
%
%%%%%%%%%%%%%%%%%%%%%%%%%%%%%%%%%%%%%%%%%%%%%%%%%%%%%%%%%%%%%%%%%%%%%%%%%%%%%%%%


%%%%%%%%%%%%%%%%%%%%%%%%%%%%%%%%%%%%%%%%%%%%%%%%%%%%%%%%%%%%%%%%%%%%%%%%%%%%%%%%
% Zur Wahl des Seitenverhältnisses bitte einen der beiden folgenden Befehle
% auskommentieren und den ausführen lassen:
%\documentclass[aspectratio=169]{beamer}
\documentclass[t,aspectratio=169]{beamer}
\usepackage[
	orientation=landscape,
	size=custom,
	width=25.4,
	height=14.2875,
	scale=0.5
]{beamerposter}
\usepackage{verbatim}
\usepackage{booktabs}
\usepackage{verbatim}
\usepackage[cache=false]{minted}
\usemintedstyle{tango}
\definecolor{codeBG}{rgb}{0.97, 0.97, 0.99}
\setminted{linenos=true, bgcolor=codeBG, mathescape=true}

\newcommand{\PraesentationSchriftgroesseSehrGross}{\fontsize{25}{38}}
\newcommand{\PraesentationSchriftgroesseGross}{\fontsize{18}{27}}
\newcommand{\PraesentationSchriftgroesseNormal}{\fontsize{14}{21}}
\newcommand{\PraesentationSchriftgroesseKlein}{\fontsize{11}{17}}
\newcommand{\PraesentationSchriftgroesseDreizeiler}{\fontsize{7}{10}}
\newcommand{\PraesentationSchriftgroesseAufzaehlungszeichen}{\fontsize{10}{8}}

\newcommand{\PraesentationAbstandAbsatz}{18pt}
\newcommand{\PraesentationPositionKorrekturOben}{-1cm}
\newcommand{\PraesentationBeispieleSchriftgroessen}{25 | 18 | 14 | 11}

%% Join Operator declarations
\usepackage{amsmath}
\usepackage{amssymb}
\usepackage{ifsym}

\def\ojoin{\setbox0=\hbox{$\bowtie$}%
    \rule[-.02ex]{.25em}{.4pt}\llap{\rule[\ht0]{.25em}{.4pt}}}

\def\leftouterjoin{\mathbin{\ojoin\mkern-5.8mu\bowtie}}
\def\rightouterjoin{\mathbin{\bowtie\mkern-5.8mu\ojoin}}
\def\fullouterjoin{\mathbin{\ojoin\mkern-5.8mu\bowtie\mkern-5.8mu\ojoin}}
\DeclareMathOperator*{\join}{\bowtie}
\DeclareMathOperator*{\leftsemijoin}{\ltimes}
\DeclareMathOperator*{\rightsemijoin}{\rtimes}
\DeclareMathOperator*{\rightantijoin}{\rhd}
\DeclareMathOperator*{\leftantijoin}{\lhd}

\input{../template/res/Praeambel.tex} % Seitenverhältnis 16:9

%%%%%%%%%%%%%%%%%%%%%%%%%%%%%%%%%%%%%%%%%%%%%%%%%%%%%%%%%%%%%%%%%%%%%%%%%%%%%%%%


%%%%%%%%%%%%%%%%%%%%%%%%%%%%%%%%%%%%%%%%%%%%%%%%%%%%%%%%%%%%%%%%%%%%%%%%%%%%%%%%
%%%%%%%%%%%%%%%%%%%%%%%%%%%%%%%%%%%%%%%%%%%%%%%%%%%%%%%%%%%%%%%%%%%%%%%%%%%%%%%%
% TUM-Vorlage: Personenspezifische Informationen
%%%%%%%%%%%%%%%%%%%%%%%%%%%%%%%%%%%%%%%%%%%%%%%%%%%%%%%%%%%%%%%%%%%%%%%%%%%%%%%%
%
% Rechteinhaber:
%     Technische Universität München
%     https://www.tum.de
% 
% Gestaltung:
%     ediundsepp Gestaltungsgesellschaft, München
%     http://www.ediundsepp.de
% 
% Technische Umsetzung:
%     eWorks GmbH, Frankfurt am Main
%     http://www.eworks.de
%
%%%%%%%%%%%%%%%%%%%%%%%%%%%%%%%%%%%%%%%%%%%%%%%%%%%%%%%%%%%%%%%%%%%%%%%%%%%%%%%%

% Für die Person anpassen:

\newcommand{\PersonTitel}{}
\newcommand{\PersonVorname}{Max}
\newcommand{\PersonNachname}{Frühauf}
\newcommand{\PersonStadt}{@Ort@}
\newcommand{\PersonAdresse}{%
    @Adresse@\\%
    @Plz@~\PersonStadt%
}
\newcommand{\PersonTelefon}{@Telefon@}
\newcommand{\PersonFax}{@Fax@}
\newcommand{\PersonEmail}{max.fruehauf@tum.de}
\newcommand{\PersonWebseite}{@Web@}

\newcommand{\FakultaetAnsprechpartner}{@Ansprechpartner@}
% Fakultät:
\newcommand{\FakultaetName}{Fakultät für Informatik}
\newcommand{\LehrstuhlName}{@Lehrstuhlname@}
% Musterdaten:
\newcommand{\EinstellungBankName}{Bayerische Landesbank}
\newcommand{\EinstellungBankIBAN}{DE10700500000000024866}
\newcommand{\EinstellungBankBIC}{BYLADEMM}
\newcommand{\EinstellungSteuernummer}{143/241/80037}
\newcommand{\EinstellungUmsatzsteuerIdentifikationsnummer}{DE811193231}

\hyphenation{} % eigene Silbentrennung                    % !!! DATEI ANPASSEN !!!
%%%%%%%%%%%%%%%%%%%%%%%%%%%%%%%%%%%%%%%%%%%%%%%%%%%%%%%%%%%%%%%%%%%%%%%%%%%%%%%%


% \renewcommand{\PersonTitel}{Dr. rer. nat.}
\newcommand{\Datum}{\today}

\renewcommand{\PraesentationFusszeileZusatz}{| Tutorium Grundlagen: Datenbanken WS 18/19}

\title{Tutorübung 7}
\author{\PersonVorname{} \PersonNachname}
\institute[]{\UniversitaetName \\ \FakultaetName}
\date[\Datum]{15. Oktober 2018}


%%%%%%%%%%%%%%%%%%%%%%%%%%%%%%%%%%%%%%%%%%%%%%%%%%%%%%%%%%%%%%%%%%%%%%%%%%%%%%%%
\input{../template/res/Anfang.tex} % !!! NICHT ENTFERNEN !!!
%%%%%%%%%%%%%%%%%%%%%%%%%%%%%%%%%%%%%%%%%%%%%%%%%%%%%%%%%%%%%%%%%%%%%%%%%%%%%%%%
\begin{document}
\setlength{\baselineskip}{\PraesentationAbstandAbsatz}
\setlength{\parskip}{\baselineskip}

%%%%%%%%%%%%%%%%%%%%%%%%%%%%%%%%%%%%%%%%%%%%%%%%%%%%%%%%%%%%%%%%%%%%%%%%%%%%%%%%
% FOLIENSTIL: Standard
% !!!ÄNDERUNG HIER:!!!
\PraesentationMasterStandard

\PraesentationTitelseite % Fügt die Startseite ein

\begin{frame}
\frametitle{Hausaufgabe 1}
\vspace{0.25cm}

\begin{multicols}{2}
	Gegeben sei die Tabelle ubahn, die strukturell dem folgenden Beispiel gleicht:
	\begin{enumerate}[a)]
		\item Geben Sie eine Anfrage an, welche für eine gegebene Station,
				beispielsweise Garching Forschungszentrum, ermittelt,
				welche \textbf{anderen} Stationen von hier erreicht werden können.
				Geben Sie diese \textbf{duplikatfrei} aus.
		% \item Ermitteln Sie die Gesamtdauer, die eine Fahrt von
		%       einer gegeben zu jeder anderen Station benötigt.
		% \item Geben Sie eine SQL Anfrage an, welche alle von Freimann
		%       in beide Richtungen rekursiv erreichbaren Stationen ermittelt.
		%       Bilden Sie hierzu zunächst die Hülle in beide Richtungen.

		%       Was ist das Problem bei der Erstellung der einfachen Hülle auf der symmetrischen Basisrelation?

	\end{enumerate}
	\vfill\columnbreak

	\begin{table}[]
		\begin{tabular}{l|l|l}
			Von                        & Nach                & Dauer        \\ \hline
			Garching Forschungszentrum & Garching            & 2            \\
			Garching                   & Garching Hochbrück  & 2            \\
			Garching Hochbrück         & Fröttmanning        & 4            \\
			Fröttmanning               & Kieferngarten       & 2            \\
			Kieferngarten              & Freimann            & 2            \\
			\( \hdots \)               & \( \hdots \)        & \( \hdots \) \\
			Odeonsplatz                & Marienplatz         & 1            \\
			\( \hdots \)               & \( \hdots \)        & \( \hdots \) \\
			Haderner Stern             & Klinikum Großhadern & 1            \\
		\end{tabular}
	\end{table}
\end{multicols}
\end{frame}

\begin{frame}[fragile]
\frametitle{Hausaufgabe 1}
\vspace{0.25cm}

\begin{multicols}{2}
	Gegeben sei die Tabelle ubahn, die strukturell dem folgenden Beispiel gleicht:
	\begin{enumerate}[a)]
		\item Geben Sie eine Anfrage an, welche für eine gegebene Station,
				beispielsweise Garching Forschungszentrum, ermittelt,
				welche \textbf{anderen} Stationen von hier erreicht werden können.
				Geben Sie diese \textbf{duplikatfrei} aus.
		% \item Ermitteln Sie die Gesamtdauer, die eine Fahrt von
		%       einer gegeben zu jeder anderen Station benötigt.
		% \item Geben Sie eine SQL Anfrage an, welche alle von Freimann
		%       in beide Richtungen rekursiv erreichbaren Stationen ermittelt.
		%       Bilden Sie hierzu zunächst die Hülle in beide Richtungen.

		%       Was ist das Problem bei der Erstellung der einfachen Hülle auf der symmetrischen Basisrelation?

	\end{enumerate}
	\vfill\columnbreak

	\begin{minted}{sql}
WITH RECURSIVE huelle (von, nach) AS (
	(SELECT von, nach FROM ubahn)
	UNION ALL
	(SELECT h.von, u.nach 
	FROM huelle h, ubahn u
	WHERE h.nach = u.von)
)
SELECT DISTINCT nach
FROM huelle
WHERE von = 'Garching Forschungszentrum'
ORDER BY von;
	\end{minted}
\end{multicols}
\end{frame}

\begin{frame}[fragile]
\frametitle{Hausaufgabe 1}
\vspace{0.25cm}

\begin{multicols}{2}
	Gegeben sei die Tabelle ubahn, die strukturell dem folgenden Beispiel gleicht:
	\begin{enumerate}[a)]
		\setcounter{enumi}{1}
		% \item Geben Sie eine Anfrage an, welche für eine gegebene Station,
		% 		beispielsweise Garching Forschungszentrum, ermittelt,
		% 		welche \textbf{anderen} Stationen von hier erreicht werden können.
		% 		Geben Sie diese \textbf{duplikatfrei} aus.
		\item Ermitteln Sie die Gesamtdauer, die eine Fahrt von
		      einer gegeben zu jeder anderen Station benötigt.
		% \item Geben Sie eine SQL Anfrage an, welche alle von Freimann
		%       in beide Richtungen rekursiv erreichbaren Stationen ermittelt.
		%       Bilden Sie hierzu zunächst die Hülle in beide Richtungen.

		%       Was ist das Problem bei der Erstellung der einfachen Hülle auf der symmetrischen Basisrelation?

	\end{enumerate}
	\vfill\columnbreak

	\begin{minted}{sql}
WITH RECURSIVE huelle(von, nach, dauer) AS (
	(SELECT von, nach dauer
	FROM ubahn
	) UNION ALL (
		SELECT u.von, h.nach,
		                  u.dauer + h.dauer
		FROM ubahn u, huelle h
		WHERE u.nach = h.von
	)
)
SELECT *
FROM huelle 
ORDER BY von;
	\end{minted}
\end{multicols}
\end{frame}

\begin{frame}[fragile]
\frametitle{Hausaufgabe 1}
\vspace{0.25cm}

\begin{multicols}{2}
	Gegeben sei die Tabelle ubahn, die strukturell dem folgenden Beispiel gleicht:
	\begin{enumerate}[a)]
		\setcounter{enumi}{2}
		% \item Geben Sie eine Anfrage an, welche für eine gegebene Station,
		% 		beispielsweise Garching Forschungszentrum, ermittelt,
		% 		welche \textbf{anderen} Stationen von hier erreicht werden können.
		% 		Geben Sie diese \textbf{duplikatfrei} aus.
		% \item Ermitteln Sie die Gesamtdauer, die eine Fahrt von
		%       einer gegeben zu jeder anderen Station benötigt.
		\item Geben Sie eine SQL Anfrage an, welche alle von Freimann
		      in beide Richtungen rekursiv erreichbaren Stationen ermittelt.
		      Bilden Sie hierzu zunächst die Hülle in beide Richtungen.

		      Was ist das Problem bei der Erstellung der einfachen Hülle auf der symmetrischen Basisrelation?
	\end{enumerate}
	\begin{minted}{sql}
WITH undir_ubahn(von, nach) AS (
	(SELECT von, nach FROM ubahn)
	UNION ALL
	(SELECT nach, von FROM ubahn)
),
	\end{minted}
	\vfill\columnbreak

	\begin{minted}{sql}
huelle(von, nach) AS (
	(SELECT von, nach FROM undir_ubahn)
	UNION
	(SELECT h.von, u.nach
	FROM huelle h, ubahn u
	WHERE h.nach = u.von)
),
zusammenhang(von, nach) AS (
	(SELECT nach, von FROM huelle)
	UNION
	(SELECT h.von, z.nach 
	FROM huelle h, zusammenhang z
	WHERE h.nach = z.von)
)
SELECT nach
FROM zusammenhang
WHERE von = 'Freimann'
ORDER BY nach;
	\end{minted}
\end{multicols}
\end{frame}
%%%%%%%%%%%%%%%%%%%%%%%%%%%%%%%%%%%%%%%%%%%%%%%%%%%%%%%%%%%%%%%%%%%%%%%%%%%%%%%%
\end{document} % !!! NICHT ENTFERNEN !!!
%%%%%%%%%%%%%%%%%%%%%%%%%%%%%%%%%%%%%%%%%%%%%%%%%%%%%%%%%%%%%%%%%%%%%%%%%%%%%%%%
